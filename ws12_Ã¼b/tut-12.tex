\input{./gbi-macros.tex}
\documentclass[12pt]{article}

%%\usepackage[blattnr=1,loesung]{thwaufgaben}
\usepackage{thwtextabbrevs}
%\usepackage{thwmathabbrevs}
\usepackage{thwmathligs}
\usepackage{german}
\usepackage{mdwlist}
\usepackage{enumerate}
\usepackage{color}
\usepackage[utf8]{inputenc}
\usepackage{graphicx}
\usepackage{amsmath}
\usepackage{amssymb}
\usepackage{wasysym}
\usepackage[paper=a4paper]{geometry}
\usepackage{graphicx}
\usepackage{booktabs}
\usepackage{xspace}

\usepackage[amsmath,thmmarks]{ntheorem}

\theoremheaderfont{\bfseries}
\theoremstyle{margin}
\usepackage{amsmath}
%\usepackage{amssymb}
%\usepackage{bold-extra}
\usepackage{checkend}
\usepackage{color}
\usepackage{graphicx}
\usepackage{picinpar}
\usepackage[alsoload=binary]{siunitx}
\usepackage{tikz}
\usetikzlibrary{arrows}
   \usetikzlibrary{automata}
   \usetikzlibrary{matrix}
\usepackage{url}

\usepackage{kbordermatrix}

\usepackage[amsmath,thmmarks]{ntheorem}

\usepackage[blue]{thwregex}
\usepackage{thwalg}
\usepackage{thwtextabbrevs}

\def\tightlist{\setlength{\itemsep}{0pt}\setlength{\parsep}{0pt}\setlength{\parskip}{0pt}}
\renewcommand{\dh}{d.\,h.\@\xspace}

\theoremheaderfont{\bfseries}
\theoremstyle{margin}

\newcommand{\thwtheoremindent}{\relax}

\newtheorem{aaaaaa}{aaaaaa}[section]
% \theoremsymbol{\ensuremath{\Diamond}}
\theorembodyfont{\upshape} \newtheorem{definition}[aaaaaa]{\thwtheoremindent Definition.}

\theorembodyfont{\upshape} \newtheorem{lemma}[aaaaaa]{\thwtheoremindent Lemma.}
\newenvironment{beweis}%
  {\begin{proofinternal}}%
  {\endofproof\end{proofinternal}}%
\newdimen\endofproofsize\endofproofsize=0.5em
\def\endofproof{~\quad\hglue\hsize minus\hsize
                  \hbox{\vrule height \endofproofsize width \endofproofsize}\par}

\def\Red#1{\textcolor{darkred}{#1}}

\begin {document}


\bibliography{../skript/gbi}

\setcounter{section}{11}

\section{Turingmaschinen}


  \subsubsection*{partielle Funktionen}
  \begin{itemize}
  \item ggf.\ noch mal kurz drauf eingehen
  \item "`wie eine normale Funktion, aber an manchen Stellen evtl.\ nicht definiert"'
  \end{itemize}

%\section{Turingmaschinen}

\subsection{Bestandteile einer Turingmaschine}
  \begin{itemize}
  \item wir betrachten nur die simpelste Variante:
    \begin{itemize}
    \item nur ein Kopf
    \item nur ein Arbeitsband
    \item keine separaten "`Spezialbänder"' für Eingaben oder Ausgaben
    \end{itemize}
  \end{itemize}

  \begin{itemize}
  \item wir haben die Arbeitsweise auf 3 Funktionen $f$, $g$, $m$
    aufgeteilt, weil man da einfacher hinschreiben kann, was ein
    Schritt ist
    \begin{enumerate}
    \item Übergang in neuen Zustand
    \item Feld mit nächstem Symbol beschriften
    \item Kopf bewegen
    \end{enumerate}
  \item weiteres Beispiel:
    \begin{center}
      \begin{tabular}{c}
        \begin{tikzpicture}[shorten >=1pt,initial text=,node distance=2cm,auto,->,>=stealth,baseline=(B.base)]
          % \node[state,initial]  (S)                       {$S$};
          \node[state,initial]  (A)          {$A$};
          % \node (nix) [right of=A] {};
          \node[state]          (B) [above right of=A] {$B$};
          \node[state]          (C) [right of=B] {$C$};
          \node[state]          (E) [below right of=A] {$E$};
          \node[state]          (D) [right of=E] {$D$};
          \path[->]
          % (S) edge              node  {$\9\io\9R$} (A)
          (A) edge              node  {$\#1\io\#XR$} (B)
          (B) edge [loop above] node  {$\#1\io\#1R$} ()
          edge              node  {$\9\io\9R$} (C)
          (C) edge [loop above] node  {$\#1\io\#1R$} ()
          edge              node  {$\9\io\#1L$} (D)
          (D) edge [loop below] node  {$\#1\io\#1L$} ()
          edge              node  {$\9\io\9L$} (E)
          (E) edge [loop below] node  {$\#1\io\#1L$} ()
          edge              node  {$\#X\io\#1R$} (A)
          ;
        \end{tikzpicture}
        \\[5mm]
        \begin{tabular}[t]{>{$}c<{$}@{\qquad}*{6}{>{$}c<{$}}}
          \toprule
          & A       & B      & C       & D       & E \\
          \midrule
          \9
          &         & C,\9,R & D,\#1,L & E,\9,L  &  \\
          \#1         & B,\#X,R & B,\#1,R& C,\#1,R & D,\#1,L & E,\#1,L \\
          \#X         &         &        &         &         & A,\#1,R \\
          \bottomrule
        \end{tabular}
      \end{tabular}
    \end{center}
    Wenn ich mich nicht vertan habe, kopiert diese TM ein Wort
    $\#1^k$ auf einem leeren Band, so dass hinterher $\cdots \blank
    \#1^k \blank \#1^k \blank \cdots$ da steht, falls man auf der
    ersten \#1 startet.
  \item Verallgemeinerung fürs Kopieren von Wörtern über $\{\#0,\#1\}$:
    Man muss sich auf dem Weg nach rechts merken, was mit \#X
    überschrieben wurde, und man muss auf dem Weg nach rechts und
    zurück sowohl \#1 als auch \#0 überlaufen.
  \end{itemize}

\subsubsection{Berechnungen}

  \begin{itemize}
  \item Wichtig: Darauf eingehen, dass es unendliche Berechnungen
    gibt --- bei TM genauso wie in Java.
  \end{itemize}

\subsubsection{Ergebnisse von Turingmaschinen}

  \begin{itemize}
  \item die Palindrommaschine aus der Vorlesung (ca Folie 32/33) sollte klar sein
  \item wir beschränken uns weitgehend auf Akzeptoren, aber wer will, kann eine TM besprechen, die eine binär dargestellte
    Zahl um 1 erhöht:

    \begin{tabular}[t]{>{$}c<{$}@{\qquad}*{4}{>{$}c<{$}}}
      \toprule
      & r & c_0 & c_1 & h \\
      \midrule
      \#0 & r,\#0,R   & c_0,\#0,L & c_0,\#1,L \\
      \#1 & r,\#1,R   & c_0,\#1,L & c_1,\#0,L \\
      \9  & c_1,\9, L & h,\9 ,R   & c_0,\#1,L & \hphantom{C,\#1,L} \\
      \bottomrule
    \end{tabular}\\

    \noindent

  \end{itemize}

\subsection{Berechnungskomplexit\"at}

  \begin{itemize}
  \item Noch mal der Hinweis aus dem Skript: Der Einfachheit halber
    wollen wir in diesem Abschnitt davon ausgehen, dass wir
    ausschließlich mit Turingmaschinen zu tun haben, die für jede
    Eingabe halten.

    In nächsten Abschnitt werden wir dann aber wieder gerade von dem
    allgemeinen Fall ausgehen, dass eine Turingmaschine für manche
    Eingaben \emph{nicht} hält.
  \item Das ist vielleicht etwas verwirrend für die Studenten. Aber
    der Formalismus für Komplexitätstheorie, bei dem alles für
    manchmal nicht haltende TM durchgezogen wird, ist auch nicht
    gerade so toll. Und man muss aufpassen.
  \item Lieber die Studenten anflehen, sie mögen doch bitte glauben,
    dass die Annahme des Immer-Haltens ok ist. Wir werden auch in der
    Vorlesung was dazu sagen.
  \end{itemize}

\subsubsection{Komplexit\"atsma\ss e}

  \begin{itemize}
  \item bei Komplexitätsmaßen üblich: \zB bei der Zeitkomplexität
    Angabe des schlimmsten Falles in Abhängigkeit von der
    Eingabegröße (und nicht für jede Eingabe einzeln).
  \item "`Auflösen"' der Rekursion $\fTime(n+2) \leq 2n+1 +
    \fTime(n-2)$ und Ergebnis $\fTime(n) \in \Oh{n^2}$ ggf.\ klar
    machen können.
  \end{itemize}

  \begin{itemize}
  \item Bitte Zusammenhänge zwischen Zeit- und
    Raumkomplexität klar machen.
  \item Um zu sehen, dass man auf linearem Platz exponentielle Zeit
    verbraten kann:
    \begin{itemize}
    \item man baue noch eine TM: auf dem Band steht anfangs eine Folge
      von Nullen. Aufgabe der TM: Solange auf dem Band nicht eine
      Folge nur aus Einsen steht, immer wieder die TM von weiter vorne
      anwenden, die die Zahl um 1 erhöht.
    \item Wenn anfangs $n$ Nullen auf dem Band stehen, dann wird
      $2^n-1$ mal die 1.~TM ausgeführt; das macht insgesamt
      offensichtlich $\geq 2^n$ Schritte.
    \item Für die, die es genauer machen wollen: \textbf{Achtung:}
      $\Th{(2^n-1)n} \not\subseteq \Oh{2^n}$, aber natürlich \zB
      $\Th{(2^n-1)n} \subseteq \Oh{3^n}$.
    \end{itemize}
  \end{itemize}

  \begin{itemize}
  \item \textbf{GANZ WICHTIG:} \emph{Niemals} von der
    Zeitkomplexität o.\,ä.~eines Problems reden.
    \begin{itemize}
    \item Algorithmen haben eine Laufzeit. Probleme nicht.
    \item \textbf{Achtung:} Der Versuch die Laufzeit eines Problems
      als die der schnellsten Algorithmen zur Lösung des Problems zu
      definieren \text{funktioniert nicht}.

      \emph{Es gibt Probleme, für die es keine schnellsten
        Algorithmen gibt.} Sondern zu jedem Algorithmus für ein
      solches Problem gibt es einen anderen, der \emph{um mehr als
        einen konstanten Faktor (!)} schneller ist.
    \end{itemize}
  \end{itemize}

\subsubsection{Komplexit\"atsklassen}

  \begin{itemize}
  \item Aus dem Skript: Wichtig:
    \begin{itemize}
    \item Eine Komplexitätsklasse ist eine Menge von Problemen ---
      und \emph{nicht} von Algorithmen.
    \item Wir beschränken uns im folgenden wieder auf
      Entscheidungsprobleme.
    \end{itemize}
  \end{itemize}

% \subsection{Unentscheidbare Probleme}
% 
%   \begin{itemize}
%   \item Als erstes noch mal den prizipiellen Unterschied zwischen
%     "`nur in Zeit $2^{2^{2^n}}$ berechenbar"' und "`überhaupt nicht
%     berechenbar"' klar machen
%   \item auch wenn man in der Praxis nie $2^{2^{2^n}}$ Zeit hat.
%   \end{itemize}
% 
% \subsubsection{Codierungen von Turingmaschinen}
% 
%  \begin{itemize}
%   \item Die im Skript beschriebene Codierung ist so gewählt, weil
%     \begin{itemize}
%     \item man leicht von einem Wort überprüfen kann, ob es eine TM
%       codiert, und
%     \item sie bequem ist, um eine TM zu simulieren, wenn man ihre
%       Codierung hat.
%     \end{itemize}
%   \item Ich habe faulerweise darauf verzichtet, auch noch zu sagen,
%     wie man bei TM-Akzeptoren die akzeptierenden Zustände
%     codiert. Man denke sich was bequemes aus.
%   \item Wer mag, kann ja eine kleine TM codieren.
%   \end{itemize}
% 
%   \begin{itemize}
%   \item Da ich bei der Unentscheidbarkeit des Halteproblems ohne
%     universelle TM auskomme, habe ich das Thema nur kurz
%     angesprochen.
%   \item Falls ich viel Zeit habe, liefere ich eine Beschreibung
%     einer universellen TM. Ansonsten:
%   \item Wer Lust hat, kann sich ja mal für die beschriebene
%     Codierung eine universelle TM überlegen. Technisch braucht man
%     nicht viel.
%   \item Falls im Tutorium Zeit ist oder entsprechende Fragen kommen,
%     kann man die universelle TM ja mal "`auf hohem Niveau"'
%     beschreiben.
%   \end{itemize}

% \subsubsection{Das Halteproblem}
% 
%   Diagonalisierung:
%   \begin{itemize}
%   \item habe ich relativ allgemein beschreiben, aber eben nur den
%     Kern.
%   \item Der Kern sollte gaaaanz klar sein: Die "`verdorbene
%     Diagonale"' $\overline{d}$ (so nenn ich das immer; wenn Sie einen
%     besseren Begriff haben \dots) unterscheidet sich von jeder Zeile
%     der Tabelle.
%   \item Die Art der Ausnutzung der Idee variiert.
%     \begin{itemize}
%     \item Manchmal weiß man, dass die Zeilen \emph{alle} Möglichkeiten
%       einer gewissen Art umfassen, dann ist also $\overline{d}$ sicher
%       nicht von der gewissen Art; es gibt also etwas, was nicht von
%       der gewissen Art ist. (\zB Komplexitätstheorie: es gibt ein
%       Problem, dass nicht in Zeit $n^2$ o.ä.\ lösbar ist) (aber wie
%       sich zeigt \zB in Zeit $n^{2+\eps}$, wenn man die Diagonale
%       vorsichtig genug verdirbt).
%     \item Manchmal, \zB bei der Überabzählbarkeit von $\R$, ist
%       $\overline{d}$ Zeuge dafür, dass die Tabelle nie vollständig
%       sein kann.
%     \item beim Halteproblem ist es noch ein bisschen anders.
%     \end{itemize}
%   \end{itemize}
% 
%   Halteproblem:
%   \begin{itemize}
%   \item Die Aussage und der Beweis müssen sitzen.
%   \item Statt konkret über TM rede ich nur noch von Algorithmen. Man
%     mache sich im Zweifelsfall klar, wie eine TM das jeweils
%     bewerkstelligen könnte.  Ich wollte hier aber keinen Workshop
%     über "`TM-Tools"' und "`TM-Libraries"' veranstalten.
%   \item Ich diskutiere nicht die Tatsache, dass $H$ als formale
%     Sprache immerhin erkennbar (also wie man auch sagt aufzählbar)
%     ist. Dafür bräuchte man universelle TM.
%   \end{itemize}
% 
% \subsubsection{Die Busy-Beaver-Funktion}
% 
%   \begin{itemize}
%   \item Aus Zeitgründen müssen wir uns hier aufs Staunen
%     beschränken. Vielleicht macht das ja den ein oder anderen
%     neugierig.
%   \item Wenn ich mich recht erinnere, hat der Wertebereich von
%     $\bb()$ übrigens die Eigenschaft, dass weder er noch sein
%     Komplement aufzählbar ist. Also noch schlimmer als $H$ \dots
%   \end{itemize}

\end{document}

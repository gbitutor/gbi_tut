\input{./gbi-macros.tex}
\documentclass[12pt]{article}

\usepackage{thwmathligs}
\usepackage{german}
\usepackage{mdwlist}
\usepackage{enumerate}
\usepackage{color}
\usepackage[utf8]{inputenc}
\usepackage{graphicx}
\usepackage{amsmath}
\usepackage{amssymb}
\usepackage{wasysym}
\usepackage[paper=a4paper]{geometry}
\usepackage{graphicx}
\usepackage{booktabs}
\usepackage{xspace}

\usepackage[amsmath,thmmarks]{ntheorem}

\theoremheaderfont{\bfseries}
\theoremstyle{margin}
\usepackage{amsmath}
\usepackage{checkend}
\usepackage{color}
\usepackage{graphicx}
\usepackage{picinpar}
\usepackage[alsoload=binary]{siunitx}
\usepackage{tikz}
\usetikzlibrary{arrows}
   \usetikzlibrary{automata}
   \usetikzlibrary{matrix}
\usepackage{url}

\usepackage{kbordermatrix}

\usepackage[amsmath,thmmarks]{ntheorem}

\usepackage[blue]{thwregex}
\usepackage{thwalg}
\usepackage{thwtextabbrevs}

\def\tightlist{\setlength{\itemsep}{0pt}\setlength{\parsep}{0pt}\setlength{\parskip}{0pt}}
\renewcommand{\dh}{d.\,h.\@\xspace}

\theoremheaderfont{\bfseries}
\theoremstyle{margin}

\newcommand{\thwtheoremindent}{\relax}

\newtheorem{aaaaaa}{aaaaaa}[section]
\theorembodyfont{\upshape} \newtheorem{definition}[aaaaaa]{\thwtheoremindent Definition.}

\theorembodyfont{\upshape} \newtheorem{lemma}[aaaaaa]{\thwtheoremindent Lemma.}
\newenvironment{beweis}%
  {\begin{proofinternal}}%
  {\endofproof\end{proofinternal}}%
\newdimen\endofproofsize\endofproofsize=0.5em
\def\endofproof{~\quad\hglue\hsize minus\hsize
                  \hbox{\vrule height \endofproofsize width \endofproofsize}\par}

\def\Red#1{\textcolor{darkred}{#1}}


\begin {document}
\setcounter{section}{9}

\section{Endliche Automaten}
\subsection{Erstes Beispiel: ein Getr\"ankeautomat}
  \begin{itemize}
  \item siehe Skript;
  \item Am Freitag in der Übung wird als weiteres Beispiel die Benutzung von
    Mealy-Automaten für einfache Codierungs- \bzw Decodierungsaufgaben
    vorkommen.
  \end{itemize}
\subsection{Mealy-Automaten}
  \begin{itemize}
  \item Man nehme den Getränkeautomaten und
    \begin{itemize}
    \item "`überlege"' sich $f^*((0,\#-), \#{R1O})$ (durch den
      Zustandsgraphen laufen)
    \item "`berechne"' $f^*((0,\#-), \#{R1O})$
    \item analog $f^{**}$
    \end{itemize}
  \item Man erarbeite die alternative Definition
    \begin{eqnarray*}
      f^{**}(z, \eps) &=& z \\
      \text{und für alle $x\in X$ und $w\in X^*$ ist\ }
      f^{**}(z, xw)   &=& z \cdot f^{**}(f(z,x),w) \\
    \end{eqnarray*}
  \item Apropos alternative Definition: In der letzten Klausur galt es per Induktion zu zeigen, dass $f^*(z,wx) = \bar{f}^*(z,xw)$.
      Die Antworten haben mich damals doch etwas traurig gestimmt. Wer nochmal eine Induktion im Tut üben möchte, kann das gerne rechnen lassen.
  \end{itemize}
  Man betrachte die folgenden Beispielautomaten:
  \begin{itemize}
  \item Getränkautomat: man mache sich klar:
    \begin{itemize}
    \item $g^*((0,\#-), \#{R1O})= \#R$
    \item $g^{**}((0,\#-), \#{R1O})= \#R$
    \item $g^{**}((0,\#-), \#{R11O})= \#{1R}$
    \end{itemize}
  \item nur ein Zustand $z$, $X=Y=\{\#a,\#b\}$
    und $g(z,\#a)=\#{b}$ und $g(z,\#b)=\#{ba}$
    \begin{itemize}
    \item wie sieht $w_1=g^{**}(z,\#a)$ aus?
    \item $w_2=g^{**}(z,w_1)$, \dots $w_{i+1}=g^{**}(z,w_i)$?
    \item was passiert mit den Längen?
    \end{itemize}
  \item $Z=\Z_5$, $X=\{\#a,\#b\}$, $Y=\{\#0,\#1\}$, bei $\#b$ gleicher
    Zustand, Ausgabe $\#0$, bei $\#a$ einen Zustand weiter, bei jedem
    5.~$\#a$ Ausgabe $\#1$, sonst Ausgabe $\#0$. Was tut der Automat?
  \end{itemize}
\subsection{Moore-Automaten}
  \begin{itemize}
  \item Die Unterschiede zwischen Moore- und Mealy-Automaten sind
    "`klein"': Abgesehen vom leeren Wort, für das ein Mealy-Automat
    keine Ausgabe liefern kann, gilt: Man kann zu jedem
    Moore-Automaten einen Mealy-Automaten konstruieren, so dass das
    $g^*$ für beide gleich ist. Und die umgekehrte Richtung von Mealy-
    zu Moore-Automaten funktioniert auch.

  \item Falls jemand fragt: Die erste Richtung von Moore zu Mealy ist
    ganz einfach: Man "`zieht die Ausgabe aus einem Zustand "`zurück"'
    zu den Eingaben an den Kanten zu diesem Zustand.

    Die umgekehrte Richtung ist ein bisschen aufwändiger, aber auch
    kein Hexenwerk; siehe
    \url{http://de.wikipedia.org/wiki/Mealy-Automat}, Abschnitt
    \emph{Zusammenhang\_mit\_Moore-Automat}.
  \end{itemize}
\subsection{Endliche Akzeptoren}
\subsubsection{Beispiele formaler Sprachen, die von endlichen Akzeptoren akzeptiert werden k\"onnen}
  \begin{itemize}
  \item \textbf{Bitte bitte bitte die akzeptierenden Zustände nur so
      nennen}, und \emph{nicht} Endzustände. Langjährige Erfahrung
    zeigt, dass das zu falschen Intuitionen führt.
  \item Man entwickele einen Akzeptor mit $X=\{\#a,\#b\}$, der alle
    Wörter akzeptiert, bei denen die Anzahl der $\#a$ durch $5$
    teilbar ist. (Anzahl der \#b ist also egal.)

    Kreis mit 5 Zuständen; bei jedem \#a eins weiter, bei jedem \#b
    Schlinge; akzeptieren bei Anfangszustand.
  \item Man entwickele einen Akzeptor mit $X=\{\#a,\#b\}$, der alle
    Wörter akzeptiert, in denen nirgends hintereinander zwei \#b
    vorkommen. Hier "`muss"' man zählen, wieviele \#b unmittelbar
    hintereinander kamen, aber nur bis $2$:

    \begin{tikzpicture}[shorten >=1pt,node distance=2cm,auto,initial text=,>=stealth]
      \node[state,initial,accepting]  (q_0)                       {$0$};
      \node[state,accepting]          (q_1) [right of= q_0] {$1$};
      \node[state]                    (q_2) [right of= q_1] {$2$};
      \path[->] (q_0) edge [loop below]      node        {$\#a$} ()
      edge [bend right] node [swap] {$\#b$} (q_1)
      (q_1) edge              node        {$\#b$} (q_2)
      edge [bend right] node [swap] {$\#a$} (q_0)
      (q_2) edge [loop below] node        {$\#a,\#b$} ()
      ;
    \end{tikzpicture}
  \item Diskussion: einfachste Version von Syntaxanalyse
  \end{itemize}
\subsection{Eine formale Sprache, die von keinem endlichen Akzeptor akzeptiert werden kann}
  \begin{itemize}
  \item Der Beweis, dass $\{\#a^k\#b^k \mid
    k\in\N_0\}$ von keinem endlichen Akzeptor erkannt werden kann, vermittelt einem
    wesentliches über endliche Automaten: Wenn ein
    hinreichend langes Wort $w$ akzeptiert wird (und das ist
    garantiert immer der Fall, wenn die Sprache unendlich ist), dann
    läuft man für ein Teilwort $v$ durch eine Schleife, und dann
    ändert mehrfaches Durchlaufen der Schleife (\bzw ganz weglassen)
    nichts am Akzeptierungsverhalten (Pumpinglemma für reguläre
    Sprachen, das kommt aber erst im dritten Semester).
  \end{itemize}

\subsection{alte Klausuraufgaben}
  \begin{itemize}
   \item Zu Akzeptoren gibt es eigenlich in fast jeder Klausur eine Aufgabe.
   \item Zu Mealy Automaten habe ich auf die Schnelle Aufgabe 3 gefunden:\\
    http://gbi.ira.uka.de/archiv/2009/k-mar10.pdf
  \end{itemize}

\end{document}

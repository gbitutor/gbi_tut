\section[R�ckblick]{Aufgabenblatt 12}
\subsection*{}
\begin{frame}
	\frametitle{Aufgabenblatt 12}
	\begin{block}{Blatt 12}
		\begin{itemize}
			\item Abgaben: 13 / 24
			\item Punkte: Durchschnitt 13,3 von 19
		\end{itemize}
		h�ufige Fehler:
		\begin{itemize}
			\item 1) Was soll bewiesen werden? nicht die Richtung verwechseln!
		\end{itemize}
   \end{block}
\end{frame}

\section[Blatt 13]{Aufgabenblatt 13}
\subsection*{}
\begin{frame}
	\frametitle{Aufgabenblatt 13}
	\begin{block}{Blatt 13}
		\begin{itemize}
			\item Abgabe: 03.02.2012 um 12:30 Uhr im Untergeschoss des Infobaus
			\item Punkte: maximal 18
		\end{itemize}
  	\end{block}
	\begin{block}{Themen}
		\begin{itemize}
	  		\item Akzeptoren
	  		\item �quivalenzrelationen
	  		\item Nerode Relationen
	 	\end{itemize}
	\end{block}
\end{frame}

\section{Probeklausur}
\subsection*{}
\begin{frame}
	\frametitle{Probeklausur}
	\begin{block}{Termin}
		\begin{itemize}
			\item Freitag den 03.02.2012 anstelle der Gbi-�bung
			\item Ort: je nach Matrikelnummer im HSaF, im HS -101 oder im HS37 (siehe GBI Seite)
      \item Teilnahme freiwillig
      \item Tutoren korrigieren die Abgaben, R�ckgabe n�chste Woche im Tutorium
      \item Ergebnis dient nur eurer Selbsteinsch�tzung
      \item Aufgaben werden von Tutoren erstellt, keine Garantie auf identische Aufgaben in der Klausur
		\end{itemize}
  	\end{block}
\end{frame}

% !TeX root = ../tut10.tex
% !TeX encoding = latin1

% ----- ab hier eigentlicher Inhalt -------------------------------------------
\section[Einstieg]{Aufwachen}
\subsection*{}
\begin{frame}{Zum Warmwerden...}
   Algorithmen-Effizienz...
    \begin{enumerate}
    \item { \only<2->{ \color{green!50!black} }
    ... wird häufig in Abhängigkeit der Eingeabelänge angegeben.
    }
    \item { \only<2->{ \color{red} }
    ... ist unabhängig von der Struktur der eingegebenen Daten.
    }
    \item { \only<2->{ \color{red} }
    ... muss für jede Rechenmaschine einzeln ermittelt werden.
    }
    \end{enumerate}

  Das O-Kalkül ...
    \begin{enumerate}
    \item { \only<3->{ \color{red} }
    ... eignet sich gut um einen Mindestaufwand anzugeben.
    }
    \item { \only<3->{ \color{green!50!black} }
    ... ist unabhängig von einfachen Faktoren.
    }
    \item { \only<3->{ \color{green!50!black} }
    ... beschreibt eine Menge von Funktionen.
    }
    \end{enumerate}

  Das $\Theta$-Kalkül...
    \begin{enumerate}
    \item { \only<4>{ \color{green!50!black} }
    ... gibt einen "`Korridor"' an, den der Algorithmus nie verlässt.
    }
    \item { \only<4>{ \color{red} }
    ... $\Theta(f(n))$ entält alle Funktionen, die auch in $O(f(n))$ enthalten sind.
    }
    \item { \only<4>{ \color{green!50!black} }
    ... ist reflexiv (Es gilt: $f(n) \in \Theta(f(n))$).
    }
    \end{enumerate}
\end{frame}

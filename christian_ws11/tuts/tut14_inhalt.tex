\section[Halbordnungen]{Halbordnungen}

\subsection*{}
\begin{frame}
  \frametitle{WDH: Definition von Äquivalenzrelationen}

	\begin{block}{Vorraussetzungen} \pause
		\begin{description}
			\item[reflexiv] $x R x$
			\item[transitiv] Aus $x R y$ und $y R z$ folgt $x R z$
			\item[symmetrisch] Aus $x R y$ folgt $y R x$
		\end{description} \pause
		Gelten alle diese Eigenschaften für alle $x,y,z \in M$, handelt es sich um eine \textbf{Äquivalenzrelation}.
	\end{block}
\end{frame}

\subsection*{}
\begin{frame}
  \frametitle{Definition einer Halbordnung}

	\begin{block}{Vorraussetzungen}
		\begin{description}
			\item[reflexiv] $x R x$
			\item[transitiv] Aus $x R y$ und $y R z$ folgt $x R z$ \pause
			\item[antisymmetrisch] Aus $x R y$ und $y R x$ folgt $x = y$
		\end{description} \pause
\begin{itemize}
	\item Gelten alle diese Eigenschaften für alle $x,y$, handelt es sich bei $R \subseteq M x M$ um eine \textbf{Halbordnung}. \pause
	\item	Wenn R Halbordnung auf Menge M ist, nennt man M eine \textbf{halbgeordnete Menge}.
\end{itemize}
	\end{block}
\end{frame}


\subsection*{}
\begin{frame}
  \frametitle{Beispiel Mengeninklusion}

	\begin{block}{Untersuchung der Mengeninklusion}
      Handelt es sich bei der Relation $\subseteq$ (Mengeninklusion) um eine \textbf{Äquivalenzrelation oder Halbordnung} auf Potenzmenge $P = 2^M$? \pause
\begin{itemize}
	\item relfexiv: $\forall A \in P$: $A \subseteq A$ \pause
	\item transitiv: $\forall A, B, C \in P$: $A \subseteq B$ und $B \subseteq C \Longrightarrow A \subseteq C$ \pause
	\item symmetrisch: $\forall A, B \in P$: $A \subseteq B \Longrightarrow B \subseteq A$
	\only<5->{ gilt nicht. \textbf{ABER:} Aus \textbf{keiner Symmetrie} folgt nicht notwendig die Antisymmetrie!
	}
	\only<6->{
	\item antisymmetrisch: $\forall A, B \in P$: $A \subseteq B$ und $B \subseteq A \Longrightarrow A = B$ (Analogie zur Mengengleichheit)
	}
\end{itemize}
	\only<7->{
	\textbf{Die Mengeninklusion ist eine Halbordnung.}
	}
  \end{block}
\end{frame}

\subsection*{}
\begin{frame}
  \frametitle{Ihr seid dran...}

	\begin{block}{Aufgabe}
	Überprüft, ob es sich bei folgenden Relationen um Halbordnungen handelt:
  \begin{itemize}
    \item  $\sqsubseteq_p$ auf $A^*$ mit $v\sqsubseteq_p w \Leftrightarrow \exists u: vu=w$ ?
      \only<2->{
      \begin{itemize}
      \item \textbf{Reflexivität}: gilt wegen  $w_1\epsilon=w_1$
      \item \textbf{Antisymmetrie}: wenn $w_1\sqsubseteq_p w_2$ und $w_2\sqsubseteq w_1$,
        dann gibt es $u_1,u_2\in A^*$ mit $w_1u_1=w_2$ und
        $w_2u_2=w_1$. Also ist $w_1u_1u_2=w_2u_2=w_1$. Also muss
        $|u_1u_2|=0$ sein, also $u_1=u_2=\epsilon$, also $w_1=w_2$.
      \item \textbf{Transitivität}: wenn $w_1\sqsubseteq_p w_2$ und $w_2\sqsubseteq w_3$,
        dann gibt es $u_1,u_2\in A^*$ mit $w_1u_1=w_2$ und
        $w_2u_2=w_3$.  Also ist $w_1(u_1u_2)=(w_1u_1)u_2=w_2u_2=w_3$,
        also $w_1\sqsubseteq w_3$.
      \end{itemize}
      }
   \only<3->{ \item $\sqsubseteq$ auf $A^*$ mit $w_1\sqsubseteq w_2 \Leftrightarrow |w_1| \leq |w_2|$ ? }
    \only<4->{
			\begin{itemize}
				\item Antisymmetrie ist verletzt.
				 \only<5->{ \item Reflexivität und Transitivität sind erfüllt. }
			\end{itemize}
		}
   \end{itemize}

  \end{block}
\end{frame}

\subsection*{}
\begin{frame}
  \frametitle{Hassediagramm}
  \begin{block}{Konstruktion}
  Zur \textbf{Veranschaulichung einer Halbordnung} lassen sich Hassediagramme folgendermaßen erstellen: \pause
    \begin{enumerate}
    \item Darstellung der Halbordnung als Graph \pause
    \item Entfernen aller reflexiven und transitiven Kanten
    \end{enumerate}
    	\end{block}
\end{frame}

\subsection*{}
\begin{frame}
  \frametitle{Extreme "`Elemente"' I}
  Sei $( M, \sqsubseteq )$ halbgeordnet und $T \subseteq M$ \pause
  \begin{block}{minimale und maximale Elemente}
    \begin{itemize}
    \item $x \in T$ heißt \textbf{minimales Element} von T, wenn es kein $y \in T$ gibt mit $y \sqsubseteq x$ und $y \neq x$. \pause
    \item $x \in T$ heißt \textbf{maximales Element} von T, wenn es kein $y \in T$ gibt mit $x \sqsubseteq y$ und $x \neq y$. \pause
    \end{itemize}
	\end{block}
	\begin{block}{kleinstes und größtes Element}
    \begin{itemize}
    \item $x \in T$ heißt \textbf{kleinstes Element} von T, wenn für alle $y \in T$ gilt: $x \sqsubseteq y$. \pause
    \item $x \in T$ heißt \textbf{größtes Element} von T, wenn für alle $y \in T$ gilt: $y \sqsubseteq x$.
    \end{itemize}
  \end{block}
  \only<6->{
  Eine Teilmenge T kann mehrere minimale (bzw. maximale) Elemente besitzen, aber nur ein kleinstes (bzw. größtes)!
  }
\end{frame}

\subsection*{}
\begin{frame}
  \frametitle{Beispiel mit Hassediagramm}
  \begin{block}{Beispiel}
    \begin{itemize}
    \item Male das Hassediagramm zur Halbordnung $(\{ \{\}, a, b, c, ab, bc, ac\}, \subseteq)$ \pause
    \item woran erkennt man Minima? \pause
    \item woran Maxima?
    \end{itemize}
	\end{block}
\end{frame}

\subsection*{}
\begin{frame}
  \frametitle{Extreme "`Elemente"' II}
  Sei $( M, \sqsubseteq )$ halbgeordnet und $T \subseteq M$ \pause
  \begin{block}{Untere und obere Schranken}
    \begin{itemize}
    \item $x \in M$ heißt \textbf{untere Schranke} von T, wenn für alle $y \in T$ gilt: $x \sqsubseteq y$. \pause
    \item $x \in M$ heißt \textbf{obere Schranke} von T, wenn für alle $y \in T$ gilt: $y \sqsubseteq x$.  \pause
    \end{itemize}
    \only<4->{
    Also: Schranken von T dürfen außerhalb von T liegen.
    }
	\end{block}
\end{frame}

\subsection*{}
\begin{frame}
  \frametitle{Extreme "`Elemente"' III}
	\begin{block}{Supremum und Infimum}
    \begin{itemize}
    \item Besitzt die Menge aller oberen Schranken einer Teilmenge T ein kleinstes Element, so heißt dies das \textbf{Supremum} von T ($sup(T)$) \pause
    \item Besitzt die Menge aller unteren Schranken einer Teilmenge T ein größtes Element, so heißt dies das \textbf{Infimum} von T ($inf(T)$) \pause
    \item \textbf{Achtung}: Existieren nicht, wenn
			\begin{itemize}
				\item überhaupt keine oberen (unteren) Schranken vorhanden \pause
				\item keine eindeutig kleinste (größte) Schranke aller oberer (unterer) Schranken
			\end{itemize}
		\end{itemize}
  \end{block}
\end{frame}

\subsection*{}
\begin{frame}
  \frametitle{Vollständige Halbordnungen}
  \begin{block}{aufsteigende Kette}
  wird definiert als
    \begin{itemize}
    \item abzählbar unendliche Folge $(x_0, x_1, x_2, ... )$ von Elementen
    \item mit Eigenschaft: $\forall i \in N_0$: $x_i \sqsubseteq x_{i+1}$
    \end{itemize}
	\end{block}
	\pause
	\begin{block}{vollständige Halbordnung}
    Eine Halbordnung heißt \textbf{vollständig}, wenn
    \begin{itemize}
    \item sie ein kleinstes Element $\bot$ hat und
    \item jede aufsteigende Kette $ x_0 \sqsubseteq x_1 \sqsubseteq x_2 \sqsubseteq ...$ ein Supremum $x_i$ besitzt
    \end{itemize}
	\end{block}
\end{frame}

\subsection*{}
\begin{frame}
  \frametitle{Vollständige Halbordnungen II}
  \begin{block}{Stetige Abbildungen auf vollständigen Halbordnungen}
    \begin{itemize}
    \item Beispiel aus dem Skript
    \item Gegeben sei:
    \item Terminalzeichenalphabet $T=\{a,b\}$, \pause
    \item $D$ die halbgeordnete Potenzmenge $D=2^{T^*}$ der Menge aller Wörter
    \item mit Inklusion als Halbordnungsrelation. \pause
    \item Elemente der Halbordnung sind also Mengen von Wörtern, d.h. formale Sprachen. \pause
    \item Kleinstes Element der Halbordnung ist die leere Menge $\emptyset$.  \pause
    \item Wie weiter vorne erwähnt, ist diese Halbordnung vollständig.
    \end{itemize}
	\end{block}
\end{frame}
\subsection*{}
\begin{frame}
  \frametitle{Vollständige Halbordnungen III}
  \begin{block}{Beweis}
    \begin{itemize}
    \item Es sei $v\in T^*$ ein Wort und $f_v:D\rightarrow D$ die Abbildung
      $f_v(L)=\{v\}L$, die vor jedes Wort von $L$ vorne $v$
      konkateniert.
    \item Behauptung: $f_v$ ist stetig.
    \item Beweis: Es sei $L_0\subseteq L_1\subseteq L_2\subseteq
      \cdots$ eine Kette und $L=\bigcup L_i$ ihr Supremum.

      $f_v(L_i)=\{ vw | w\in L_i \}$, also $\bigcup_i f_v(L_i)= \{ vw | \exists i\in N_0: w\in L_i\} = \{v\}\{w | \exists i \in N_0: w\in L_i\}$ $=\{v\}\bigcup_i L_i = f(\bigcup_i L_i)$.
    \item analog für Konkatenation von rechts
    \end{itemize}
	\end{block}
\end{frame}

\section[Ordnungen]{Ordnungen}
\subsection*{}
\begin{frame}
  \frametitle{Totale Ordnung}

	\begin{block}{Definition}
	Relation $R \subseteq M x M$ ist eine Ordnung oder genauer \textbf{totale Ordnung}, wenn
\begin{itemize}
	\item R Halbordnung ist \pause
	\item und gilt: $\forall x ,y \in M$: $x R y \vee y R x$
\end{itemize}
	\end{block}
\pause
	\begin{block}{Anmerkungen}
\begin{itemize}
	\item $\Rightarrow$ : Es gibt keine unvergleichbaren Elemente.
\end{itemize}
	\end{block}
\end{frame}

\subsection*{}
\begin{frame}
  \frametitle{Totale Ordnung}

	\begin{block}{Beispiele}
\begin{itemize}
	\item $( N_0 , \leq )$ \pause
	\item $( \{a, b\}^*, \sqsubseteq_1 )$ mit $w_1 \sqsubseteq_1 w_2$ "`wie im Wörterbuch"' \pause
	\item $( \{a, b\}^*, \sqsubseteq_2 )$ mit $w_1 \sqsubseteq_2 w_2$ genau dann, wenn \pause
		\begin{itemize}
			\item entweder $|w_1| < |w_2|$ \pause
			\item oder $|w_1| = |w_2|$ und $w_1 \sqsubseteq_1 w_2$ gilt
		\end{itemize}
\end{itemize}
	\end{block}
\end{frame}

\subsection*{}
\begin{frame}
  \frametitle{Ihr seid dran...}

	\begin{block}{Beispiele für $\sqsubseteq_1$:}
      \begin{itemize}
        \item  Warum ist $aa \sqsubseteq_1 aabba$? \pause
        \item  Warum ist $aa \sqsubseteq_1 bba$? \pause
        \item  Warum ist $aaaaa \sqsubseteq_1 bba$? \pause
        \item  Warum ist $aaaab \sqsubseteq_1 aab$? \pause
      \end{itemize}
  \end{block}
  \begin{block}{Beispiele für $\sqsubseteq_2$:}
      \begin{itemize}
        \item  Warum ist $aa \sqsubseteq_2 aabba$? \pause
        \item  Warum ist $aa \sqsubseteq_2 bba$? \pause
        \item  Warum ist $bba \sqsubseteq_2 aaaaa$? (vergleiche $\sqsubseteq_1$!)\pause
        \item  Warum ist $aab \sqsubseteq_2 aaaab$? (vergleiche $\sqsubseteq_1$!)
      \end{itemize}
	\end{block}
\end{frame}

\subsection*{}
\begin{frame}
  \frametitle{Bleibt dran...}

	\begin{block}{Aufgabe}
		Relation $\sqsubseteq_p$ auf $ \{a, b\}^* $ eine totale Ordnung?
	\end{block}
	\only<2->{
	\begin{block}{Lösung}
		Es handelt sich um eine Halbordnung, allerdings mit unvergleichbaren Element wie z.B. $a, b$.
		Daher ist die Relation $\sqsubseteq_p$ \textbf{keine} totale Ordnung.
	\end{block}
	}
\end{frame}

\section[Klausur]{Klausur am 01.03.11}
\subsection*{}
\begin{frame}
	\frametitle{Klausur am Dienstag, 1. März '11}
	\begin{block}{17:00 - 19:00}
		\begin{itemize}
			\item Aufgaben: 5 - 8
			\item Gesamtpunktzahl: ?? (zum Bestehen: ??)
			\item Anmeldebeginn: schon möglich
			\item Anmeldeende: 26. Februar
			\item Abmeldeende: 28. Februar
		\end{itemize}
  	\end{block}
  	\textit{Alle Angaben sind wie immer ohne Gewähr.}
\end{frame}

\subsection*{}
\begin{frame}
  \frametitle{Themen}
	\begin{block}{Themen}

		\begin{itemize}
		\item siehe Vorlesungshinweis XOR alles was behandelt wurde
	  	\item z.B. :\pause
		  	\item Mengenlehre, \textbf{Abbildungen}, Aussagenlogik, Quantoren, Wörter,
		  	\item Palindrome, \textbf{Formale Sprachen}, Grammatiken,
		  	\item Zahlensysteme, Huffman-Codes,
		  	\item \textbf{Graphen}, Adjazenzliste, Adjazenzmatrix, Wegematrix,
		  	\item Mealy- / Moore- \textbf{Automaten}, Akzeptor, Regulärer Ausdruck,
		  	\item \textbf{Äquivalenzrelationen}, Nerode-Relation, Ordnungen
	 	\end{itemize}
	\end{block}
\end{frame}

\subsection*{}
\begin{frame}
   \frametitle{Anmerkung}
	\begin{block}{WICHTIG!}
  	\begin{itemize}
			\item Meldet Euch \textbf{rechtzeitig} zur Klausur an!
		\end{itemize}
	\end{block}

\end{frame}

\section{Grammatiken}
\subsection*{}
\begin{frame}
	\frametitle{Aufgabe}
	Gib zu den folgenden Sprachen $L_1$ ,$L_2$ jeweils eine Grammatik höchstmöglichen Typs 	an (das heißt Typ i mit i möglichst groß aus {0,1,2,3}), welche die Sprache erzeugt.
	\begin{enumerate}
		\item[(a)] $L_1=\{a^m (bc)^{2m}| m> 0 \}$
		\item[(b)] $L_2=\{a^n b^m c^m d^n | m>=0, n>=1 \}$
	\end{enumerate}
	\visible<2->{
	Lösung: \\
	\begin{enumerate}
		\item[(a)] $G=(\Sigma ,N,P,A)$ mit $\Sigma=\{a,b,c\}, N=\{A\}$ und $P = \{A \rightarrow aAbcbc|abcbc\}$ \pause
		\item[(b)] $G=(\Sigma, N,P,A)$ mit $\Sigma=\{a,b,c,d \} N=\{ \}$ und \\
		$P=\{ A \rightarrow ad|aAd|aBd$ \\
		$  B \rightarrow bBc|bc \}$
	\end{enumerate}}
\end{frame}

\subsection*{}
\begin{frame}
	\frametitle{Aufgabe}
	Die hawaiianische Sprache kennt nur die folgenden Buchstaben:
	\begin{itemize}
		\item die Vokale a, e, i, o, u
		\item die Konsonanten h, k, l, m, n, p, w
	\end{itemize}
Es gelten dabei folgende Regeln:
Ein Wort beginnt mit einem Konsonanten oder einem Vokal. Auf einen Konsonanten muss
mindestens ein Vokal folgen. Es können beliebig viele Vokale aufeinander folgen. Konsonanten
dürfen nicht am Ende eines Wortes stehen. Ein Wort hat mindestens einen Buchstaben.
	\small
	\begin{enumerate}
		\item[(a)] Gib eine Grammatik des Typs 2 an, die diese Sprache erzeugt. \\
					\footnotesize
					\textbf{Hinweis:} Es gibt hier auch eine Typ-3 Grammatik, aber diese ist recht umfangreich und daher als Lösung nicht sinnvoll.
					\small
		\item[(b)] Erzeuge mittels der Grammatik aus (a) das Wort kaiulani.
		\item[(c)] Erstelle einen regulären Ausdruck für die hawaiianische Sprache.
		\item[(d)] Erstelle einen Akzeptor, der die hawaiianische Sprache akzeptiert.
	\end{enumerate}
\end{frame}

\subsection*{}
\begin{frame}
	\frametitle{Lösung}
	\begin{enumerate}
		\item[(a)] $G = (\Sigma, N, P, S)$ \\
		$\Sigma=\{a,e,i,o,u,h,k,l,m,n,p,w\}$ \ $N=\{S,V,K \}$ \\
		$P=\{ S \rightarrow V|VS|KV|KVS $ \\
				$ V \rightarrow a|e|i|o|u$ \\
				$ K \rightarrow h|k|l|m|n|p|w \}$
		\pause
		\item[(b)] $S \Rightarrow KVS \Rightarrow kVS \Rightarrow kaS \Rightarrow kaVS \Rightarrow kaiS \Rightarrow kaiVS \Rightarrow kaiuS \Rightarrow kaiuKVS \Rightarrow kaiulVS \Rightarrow kaiulaS \Rightarrow kaiulaKV \Rightarrow kaiulanV \Rightarrow kaiulani$
		\pause
		\item[(c)] $((a|e|i|o|u)|(h|k|l|m|n|p|w)(a|e|i|o|u)) 
((a|e|i|o|u)|(h|k|l|m|n|p|w)(a|e|i|o|u))*$
	\end{enumerate}
\end{frame}



\section{Abschluss}
% Studis anzuregen darüber nachzudenken, ob sie wirklich alles wissen, ansonsten nachlesen oder fragen nachträglich stellen, dann kann in der nächsten Woche nochmal drauf eingegangen werden
\subsection*{}
\begin{frame}
	\frametitle{Zum Schluss...}
	\begin{block}{Was ihr nun wissen solltet!}
	\begin{itemize}
	  	\visible<2->{\item Wie unterscheiden sich Äquivalenzrelation und Halbordnung? Was sind typische Beispiele?}
		\visible<3->{\item Warum Hassediagramme? Welche "`extremen"' Elemente treten bei Halbordnungen auf?}
		\visible<4->{\item Was besagt eine totale Ordnung?}
		\visible<5->{\item Meldet euch bitte für Schein und KLAUSUR an!\\
						Solange ihr euch nicht für den Schein anmeldet, kann er euch auch nicht eingetragen werden.}
    \end{itemize}
   	\end{block}

	\visible<6->{
	\begin{block}{Ihr wisst was nicht?}
		Stellt \textbf{jetzt} Fragen!
	\end{block}}
\end{frame}

% ----- ab hier eigentlicher Inhalt -------------------------------------------
\section[Einstieg]{Aufwachen}
\subsection*{}
\begin{frame}{Zum Warmwerden...}
  Ein ungerichteter Graph $U$...
    \begin{enumerate}
    \item { \only<2->{ \color{red} }
    ... kann eine Einbahnstraße modellieren
    }
    \item { \only<2->{ \color{red} }
    ... wurde definiert als $U=(N,T,S,P)$
    }
    \item { \only<2->{ \color{green!50!black} }
    ... hat ausschließlich symmetrische Kanten
    }
    \end{enumerate}

  Ein Pfad...
    \begin{enumerate}
    \item { \only<3->{ \color{green!50!black} }
    ... wird als Liste $p=(v_0,\ldots ,v_n)\in V^{(+)}$ angegeben
    }
    \item { \only<3->{ \color{green!50!black} }
    ... mit $v_0=v_n$ heißt geschlossen oder Zyklus
    }
    \item { \only<3->{ \color{red} }
    ... hat die Länge $|p|$
    }
    \end{enumerate}

  Ein Huffman Code $C$...
    \begin{enumerate}
    \item { \only<4->{ \color{green!50!black} }
    ... ist immer präfixfrei.
    }
    \item { \only<4->{ \color{red} }
    ... ist immer eindeutig.
    }
    \item { \only<4->{ \color{red} }
    ... kann definiert sein als $C: \{a,b,c\} \rightarrow \{0.1\}$* mit $C(a)=00, C(b)=010, C(c)=001$.
    }
    \item { \only<4->{ \color{red} }
    ... codiert selten vorkommende Symbole durch kurze Wörter.
    }
    \item { \only<4->{ \color{green!50!black} }
    ... lässt sich mittels eines Baumes bestimmen.
    }
    \end{enumerate}
\end{frame}

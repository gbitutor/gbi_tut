%%TODO: 1.3 unendliche mengen beschreiben,
%%1.4 wie kommt man auf 16: 2^2 eingaben, verknüpfung also durch 4 bit definiert, sprich für jede eingabe eine ausgabe, sprich 2^4 mögliche verknüpfungen
%%%beispiele für verknüpfungen: nicht A, und, oder, xor, nor, ..., 0, 1

\section[Organisation]{Organisatorisches}
% \subsection*{}
\subsection*{}
\begin{frame}
\frametitle{Euer Tutor}
        \begin{block}{Kontakt}
                \begin{description}
                        \item[Kontakt:] gbi-tutor@gmx.de
                        \item[Homepage:] \url{http://gbi-tutor.blogspot.com}
                \end{description}
        \end{block}
\end{frame}
\section[Blatt 1]{Aufgabenblatt 1}
\subsection*{}
\begin{frame}
	\frametitle{Ein Blick zurück}
	\begin{block}{etwas Statistik}
		\begin{itemize}
			\item 23 Abgaben, weiter so!
			\item durchschnittliche Punktzahl: $14,7$/$20$ Punkten 
		\end{itemize}
	\end{block}

	\begin{block}{häufige Fehler...}
		\begin{itemize}
			\item[1.1:] äquivalente Ausdrücke sind nicht $=$, besser $\equiv$ \pause
			\item[1.1:] bei Wahrheitstabellen immer auch Wert der Eingabevariablen angeben \pause
			\item[1.4:] denkt an die Klammern, selbst wenn in der Vorlesung Vorrangigkeit der Operatoren definiert ist!
		 \end{itemize}
	\end{block}
\end{frame}

\section[Blatt 2]{Aufgabenblatt 2}
\subsection*{}
\begin{frame}
        \frametitle{Aufgabenblatt 2}
        \begin{block}{Blatt 2}
					\begin{itemize}
						\item Abgabe: 04.11.2011 um 12:30 Uhr im Untergeschoss des Infobaus
						\item Punkte: maximal 20
					\end{itemize}
        \end{block}
        \begin{block}{Themen}
        \begin{itemize}
          \item Prädikatenlogik 
        	\item Wörter
        	\item Vollständige Induktion
        	\item Mengen
        \end{itemize}
        \end{block}
\end{frame}

\subsection{Prädikatenlogik}
\begin{frame}
	\frametitle{Quantoren}
	\begin{block}{$\exists$ und $\forall$}
	Eine nützliche Notation, um zu unterscheiden, ob wir Aussagen für alle Elemente oder nur für eines machen sind Quantoren. Die gebräuchlichsten sind:
	\begin{itemize}
		\item[$\exists$] Existenzquantor (lies: "`Es existiert"')
		\item[$\forall$] Allquantor (lies: "`Für alle"')
	\end{itemize}
	Bei den Quantoren kommt es auf die Reihenfolge an! Sie dürfen niemals hinter eine Formel stehen!
	\end{block}
	\only<2->{
	\begin{block}{Einige Fragen...}
	\begin{itemize}
	\item Welche der beiden Formeln ist gemeint? \\ $\forall y \exists x : y > x$ \only<2>{oder $\exists y \forall x : y>x$}
	\item Gilt $(\exists x A(x)) \land (\exists x B(x))\equiv \exists x: A(x) \land B(x)$? \only<3>{Nein!}
	\end{itemize}
	\end{block}
	}
\end{frame}

\subsection{Definitionen}
\begin{frame}
	\frametitle{Aus der Vorlesung...}
	\begin{block}{Alphabet}
	Ein \textbf{Alphabet} ist eine endliche Menge von Zeichen.
	\end{block}

	\pause
	\begin{block}{Wort}
		\begin{itemize}
			\item Ein \textbf{Wort} w über einem Alphabet A ist eine \textbf{Folge von Zeichen} aus A.	\pause
			\item formal: surjektive Abbildung $w : \mathbb G_n \rightarrow A$ wobei $\mathbb G_n = \left\{i \in \mathbb N_0 | 0 \leq i < n\right\}$

		\end{itemize}
	\end{block}
\end{frame}

\begin{frame}
	\frametitle{Aus der Vorlesung...}
	\begin{block}{Menge aller Wörter}
	Die Menge der Wörter der Länge \textbf{n} wird bezeichnet mit $A^n$. Die Menge aller Wörter $A^*$ ist definiert als $A^* = \bigcup^{\infty}_{i=0} A^i$.
	\end{block}

	\only<2->{
	\begin{block}{Ihr seid dran...}
		\begin{itemize}
			\item Gegeben: Alphabet $A = \{a, b\}$ Gesucht: $A^*$
			\only<3->{ \item $A^* = \{a, b, aa, ab, ba, bb, aaa, ...\}$
				\item Beachtet: $\forall$ Alphabete A ist das \textbf{leere Wort} $\epsilon
			\in A^*$.}
		\end{itemize}
	\end{block}}
\end{frame}

\begin{frame}
	\frametitle{Konkatenation}
	\begin{block}{$A^*$}
		\begin{itemize}
			\item Gegeben: Alphabet $A = \{a, b\}$, Gesucht: $A^*$
			\item $A^* = \{a, b, aa, ab, ba, bb, aaa, ...\}$
		\end{itemize}
	\end{block}

	\begin{block}{$w^n$}
		\begin{itemize}
			\item Gegeben: Wort $w = ab$, Gesucht: $w^n$ \pause
			\item $w^n = ab \cdot (w^{n-1})$
		\end{itemize}
	\end{block}
\end{frame}

\subsection{Mengenlehre}
\begin{frame}
	\frametitle{Mengenlehre}
	\begin{block}{Indexmengen}
		\begin{itemize}
			\item Es sei gegeben $\mathbb G_n = \left\{i \in \mathbb N_0 | 0 \leq i < n\right\}$
			\uncover<2->{\item Was ist $\bigcup^{\infty}_{i=0} \mathbb G_i$ ?}
			\uncover<3->{ Es ist $\mathbb N_0$.} \uncover<4->{\item Wie beweist man das?}
		\end{itemize}
	\end{block}

	\uncover<5->{
	\begin{block}{Mengengleichheit}
		\begin{itemize}
			\item Wie beweist man allgemein, das zwei Mengen $M_1$ und $M_2$ gleich sind?
			\uncover<6->{ \item	Man zeigt, dass
										\begin{enumerate}
											\item $M_1 \subseteq M_2$
											\uncover<7->{ \item $M_2 \subseteq M_1$}
										\end{enumerate}
							 }
		\end{itemize}
	\end{block}}
\end{frame}

\begin{frame}
	\frametitle{Mengenlehre}
	\begin{block}{Mengeninklusion}
		\begin{itemize}
			\item Wie beweist man $M_1 \subseteq M_2$?
			\uncover<2->{\item Man zeigt, dass $\forall x \in M_1 : x \in M_2$ }
		\end{itemize}
	\end{block}

	\uncover<3->{
	\begin{block}{Ihr seid dran...}
  	\begin{itemize}
			\item Es sei $\mathbb G_n = \left\{i \in \mathbb N_0 | 0 \leq i < n\right\}$.\\
				Zeigt nun:
						\begin{displaymath}
                        	\mathbb N_0 = \bigcup^{\infty}_{i=0} \mathbb G_i
                        \end{displaymath}
  	\end{itemize}
  \end{block}}
\end{frame}

\begin{frame}
	\frametitle{Mengenlehre}
		\begin{block}{$\subseteq$} \pause
		Wähle ein beliebiges $n \in \mathbb N_0$. Dann gilt nach Definition von $\mathbb G_{n+1}$: $n \in \mathbb G_{n+1}$ und somit auch $n \in \bigcup^{\infty}_{i=0} \mathbb G_i$.
		\end{block}
	 \pause
		\begin{block}{$\supseteq$}  \pause
		Laut Definition enhält $\mathbb G_i$ nur Elemente aus $\mathbb N_0$. Somit $\mathbb G_i \subseteq \mathbb N_0$.
  	\end{block}
\end{frame}


\section[VI]{Vollständig Induktion}
\subsection*{}
\begin{frame}
	\frametitle{Vollständige Induktion tut nicht weh...}
	\begin{block}{Ihr seid dran...}
		\begin{itemize}
			\item \only<1>{Wer}\only<2>{Ihr} kennt das Beweisverfahren der vollständige Induktion \only<1>{?}\only<2>{...} \\
				\visible<2->{Erklärt den "`Unwissenden"' das Verfahren...}
			\item \only<1>{Wer}\only<2>{Ihr} kennt das Verfahren nicht\only<1>{?}\only<2>{...} \\
		 		\visible<2->{Hört gespannt zu...}
		 \end{itemize}
	\end{block}
\end{frame}

\subsection*{}
\begin{frame}
	\frametitle{Beweisverfahren der vollständigen Induktion}
	\begin{block}{Die Theorie}
		Der Beweis erfolgt in folgenden Schritten:
		\begin{enumerate}
			\item Induktionsanfang: Die Aussage wird für $n=n_0$ gezeigt \pause
			\item Induktionsvorraussetzung/-annahme: Die Aussage sei für \textbf{ein}
			beliebiges  $n$ wahr. \pause
			\item Induktionsschluss/-schritt: Aus dem Schluss von $n$ auf $n+1$ (in der Regel mit Hilfe der IV) folgt, dass die Aussage für alle natürlichen Zahlen $n>n_0$ gilt.
		\end{enumerate}
	\end{block}

	\begin{block}{Ein Beispiel:}
	Beweise durch vollständig Induktion: 
	 $1+3+5+...+(2n-1)=n^2$
	\end{block}
	
\end{frame}

\begin{frame}
	\frametitle{Beweisverfahren der vollständigen Induktion}
	
	\begin{block}{Ein Beispiel:}
	Beweise durch vollständig Induktion $1+3+5+...+(2n-1)=n^2$
	
	\begin{itemize}
		\item[IA] $n=1$: $1=1^2=1$ ist erfüllt
		\item[IV]	$1+3+5+...+(2n-1)=n^2$ gilt für ein $n \in \mathbb{N}$
		\item[IS] \begin{equation}
				\begin{split}
					 1+3&+5+...+(2(n+1)-1) \nonumber\\
					&= 1+3+5+...+(2n+1) \nonumber\\
					&=1+3+5+...+(2n-1)+(2n+1) \nonumber\\
					&\overset{IV}{=} n^2+2n+1 \nonumber\\
					&=(n+1)^2
				\end{split}
			\end{equation}
    \end{itemize}
    \end{block}
	
\end{frame}

\subsection*{}
\begin{frame}
	\frametitle{Ein etwas komplizierteres Beispiel...}
	\only<1>{
	\begin{block}{Ihr seid dran...}
	Es sei $q\in \mathbb N_0$ und $q\geq 2$:
		$s_0 = 1 $\\
    $\forall k\in \mathbb N_0:\; s_{k+1} = s_k + q^{k+1}$ \\
 	Beweise durch vollständige Induktion:
   $\forall k\in \mathbb N_0:\; s_k = \frac{q^{k+1}-1}{q-1}$
  \end{block}}

  \only<2->{
  \begin{block}{Lösung}
  \begin{itemize}
	  \item[IA:] $k=0$: $\frac{q^{0+1}-1}{q-1} = \frac{q-1}{q-1} = 1 = s_0$
	  \item[IV:] $s_k = \frac{q^{k+1}-1}{q-1}$ gelte für ein $n$
	  \small
	  \item[IS:]
      \begin{align*}
        s_{k+1} &= s_k + q^{k+1} \text{\qquad nach Definiton} \\
        &= \frac{q^{k+1}-1}{q-1}+ q^{k+1} \text{\qquad nach Induktionsannahme} \\
        &= \frac{q^{k+1}-1+ (q-1)q^{k+1}}{q-1} \\
        &= \frac{q^{k+1}-1+ q*q^{k+1} - q^{k+1}}{q-1} \\
        &= \frac{q^{k+2}-1}{q-1}
				\end{align*}
  \end{itemize}
	\end{block}}
\end{frame}

\section{Abschluss}
% Studis anzuregen darüber nachzudenken, ob sie wirklich alles wissen, ansonsten nachlesen oder fragen nachträglich stellen, dann kann in der nächsten Woche nochmal drauf eingegangen werden
\subsection*{}
\begin{frame}
	\frametitle{Zum Schluss...}
	\begin{block}{Was ihr nun wissen solltet!}
	\begin{itemize}
		\visible<2->{\item Wie beweise ich Mengengleichheit?}
		\visible<3->{\item Was ist das Beweisverfahren der vollständigen Induktion?}
		\visible<4->{\item Was kann ich alles tolles damit anstellen?}
		\visible<5->{\item Wie kann ich meinen Tutor bei der Korrektur meines Übungsblattes positiv beeinflussen?}
	\end{itemize}
	\end{block}

	\visible<6->{
	\begin{block}{Ihr wisst was nicht?}
		Stellt \textbf{jetzt} Fragen!
	\end{block}}
\end{frame}

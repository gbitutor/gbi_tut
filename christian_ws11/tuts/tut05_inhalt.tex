% ----- ab hier eigentlicher Inhalt -------------------------------------------


\section[Kontextfreie Grammatiken]{Kontextfreie Grammatiken}
\subsection*{}
\begin{frame}
	\frametitle{Aus der Vorlesung:}
	\begin{block}{Definition 1}
		\begin{itemize}
			\item f�r alle Alphabete $A$ und alle $x\in A$ Funktionen $N_x: A^* \rightarrow \mathbf{N_0}$, die wie folgt festgelegt sind:
			\begin{itemize}
				\item $N_x(\epsilon) = 0 $
				\item $\forall y\in A: \forall w\in A^*: N_x(yw) = \begin{cases}
						1+N_x(w) & \textnormal{ falls } y= x\\
      			N_x(w) & \textnormal{ falls } y\not= x\\
					\end{cases}$
			\end{itemize}
		\item Was gibt $N_x(w)$ an?
	  \end{itemize}
	\end{block}
	\only<2>{
	\begin{block}{Definition 2}
		\begin{itemize}
			\item Kont-fr. Grammatik $G=(N,T,S,P)$, wobei
			\begin{itemize}
				\item N Nichtterminalsymbole
				\item T Terminalsymbole
					\begin{itemize}
						\item und $ N\cap T=\emptyset $
	  			\end{itemize}
	  		\item S Startsymbol ($ S \in N $)
	  		\item P Produktionen ($ P \subseteq N \times V^* $ und $ V = N \cup T $)
	  	\end{itemize}
	  \end{itemize}
	\end{block}
	}
\end{frame}
\begin{frame}
	\frametitle{Beispiele}
	\begin{block}{Beispiel 1}
		\begin{itemize}
			\item $ G=(\{ X \} , \{ a,b \},X, \{ X \rightarrow \epsilon \mid aX \mid bX \} ) $
			\item Was kann mit dieser Grammatik erstellt werden? \pause
			
		\begin{itemize} 
          \item alle W�rter �ber $A=\{a,b\}$ \pause
          \item also $L(G)= \{ a,b\} ^*$
	  	\end{itemize}
	  \end{itemize}
	\end{block} \pause
	\begin{block}{Beispiel 2}
		\begin{itemize}
			\item Gibt es eine Grammatik mit $L(G)= \emptyset $ ? \pause
			\item Ja \pause
			 \begin{itemize}
				\item $P= \{ X \rightarrow X \} \vee P= \{ \} $ \pause 
				\item aber leeres Alphabet ($T= \{ \} $) nicht zul�ssig 
			\end{itemize}
		\end{itemize}
	\end{block}
	
\end{frame}
\begin{frame}
	\frametitle{mehr Beispiele}
	\begin{block}{weiteres (einfaches) Beispiel}
		\begin{itemize}
			\visible<2->{\item $G=(\{X\}, \{ (, )\}, X, \{X \rightarrow XX \mid ( X ) \mid \epsilon \}$}
			\visible<3->{\item Beispielableitungen:
    		\begin{itemize}
    			\item $X \Rightarrow (X) \Rightarrow ((X)) \Rightarrow (((X))) \Rightarrow ((((X)))) \Rightarrow (((()))) $ oder}
			    \visible<4->{\item $X \Rightarrow XX \Rightarrow XXX \Rightarrow XXXX \Rightarrow XXXXX \Rightarrow (X)XXXX \Rightarrow ... \Rightarrow(X)(X)(X)(X)(X) \Rightarrow ()()()()()$ }
			  \end{itemize}
			\visible<5->{\item Unterschied zu $G=(\{X\}, \{ (,)\}, X, \{X \rightarrow (X)X \mid \epsilon \})$?}
	  \end{itemize}
	\end{block}
\end{frame}

\begin{frame}
	\frametitle{Aufgaben f�r euch}
	\begin{block}{Aufgabe 1}
		\begin{itemize}
			\item Sei $ T=\{ a,b \} $. Erstellt eine Grammatik, in der alle W�rter �ber	T, die {\bf baa} enthalten, vorkommen! 
			\visible<2->{ \item $G=(\{X,Y\},T,X,\{X \rightarrow Y \{baa\} Y, Y
			\rightarrow aY \mid bY \mid \epsilon\}$}
	  \end{itemize}
	\end{block}
\end{frame}



\section[Relationen]{Relationen}
\subsection*{}
\begin{frame}
	\frametitle{Relationen -- �berblick}
	\begin{block}{Relationen anschaulich}
		Durch Relationen werden Elemente einer oder mehrerer Mengen in Beziehung zueinander gesetzt:
		\begin{itemize}
			\item praktisch jede allt�gliche Aussage enth�lt Relationen \pause
			\item Beispiel: 'Das Haus hat vier Au�enw�nde'
		\end{itemize}
	\end{block}
	\pause
	\begin{block}{Wozu brauchen wir das?}
		In der Informatik werden Relationen zur Modellierung von Systemen ben�tigt: \pause
		\begin{itemize}
			\item Relationen sind Grundlage der verschiedenen Diagramme der Unified
			Modeling Language \pause
			\item die graphische Darstellung von Relationen ergibt Graphen
		\end{itemize}
	\end{block}

\end{frame}

\begin{frame}
	\frametitle{Relationen mathematisch}
	\begin{block}{Definitionen - aus dem 1. Tutorium}
		\begin{itemize}
			\item Das karthesische Produkt zweier Mengen ist definiert als $A \times B := \left\{(a, b)|a \in A, b \in B\right\}$
			\item $R \subseteq A \times B$ hei�t Relation
	  	\end{itemize}
	\end{block}
	\pause
	\begin{block}{Definition}
	\begin{itemize}
		\item Eine Relation $R$ bezieht sich auf zwei Grundmengen $M_1$,$M_2$ und es gilt $R \subseteq M_1\times M_2$. \pause
		\item Eine Relation $R$ hei�t homogen, wenn $M_1=M_2$ gilt.
	\end{itemize}
	\end{block}
\end{frame}

%

\begin{frame}
	\frametitle{Relationen}
	\begin{block}{Definition: Produkt}
		Sind $R \subseteq M_1 \times M_2$ und $S \subseteq M_2 \times M_3$ zwei Relationen, dann hei�t \pause
		\begin{itemize}
			\item $S \circ R = \{(x,z) \in M_1 \times M_3 \mid \exists y \in M_2 : (x,y) \in R \wedge (y,z) \in S \} $ das {\em Produkt der Relationen S und R} \pause
			\item $Id_M = \{(x,x) \mid x \in M \}$ hei�t die {\em identische Abbildung}
		\end{itemize}
	\end{block}
	\pause
	\begin{block}{Definition: Potenz}
		Sei $R \subseteq M \times M$ eine {\em bin�re} Relation, dann hei�t
		\begin{itemize}
			\item $R^i$ {\em die i-te Potenz} von R und ist definiert als: \pause
			\begin{itemize}
				\item $R^0=Id_M$ \pause
				\item $\forall i \in \mathbb N_0: R^{i+1}=R \circ R^i$
			\end{itemize}
		\end{itemize}
	\end{block}
\end{frame}

\section[Reflexiv-transitive H�lle]{Reflexiv-transitive H�lle}
\subsection*{}
\begin{frame}
	\frametitle{Reflexiv-transitive H�lle}

	\begin{block}{m�gliche Attribute homogener Relationen}
		\begin{description}
			\item[reflexiv] $x R x$
			\item[transitiv] Aus $x R y$ und $y R z$ folgt $x R z$
			\item[symmetrisch] Aus $x R y$ folgt $y R x$
		\end{description} \pause
		Gelten alle diese Eigenschaften, handelt es sich um eine �quivalenzrelation.
	\end{block}
		\pause
	\begin{block}{Definition}
		Die sogenannte reflexiv-transitive H�lle einer Relation R ist
		\begin{itemize}
			\item $R^* = \bigcup_{i=0}^{\infty} R^i$
		\end{itemize}
	\pause
		Sie ist die Erweiterung der Relation um die Paare, die notwendig sind um Reflexivit�t und Transitivit�t herzustellen.
	\end{block}

\end{frame}

\begin{frame}
	\frametitle{Reflexiv-transitive H�lle}
	\begin{block}{Anschauliches Beispiel: StudiVZ}
		\begin{itemize}
			\item $R \subseteq M \times M $ sei die "`ist-befreundet-mit"'-Relation.
			\item $M = \{ Gertrud, Holger, Lars, Katja, Martin, Nina \} $ %alphabetische Anf�nge == leichter merkbar
			\item $R = \{ (Martin,Holger), (Lars,Katja), (Nina,Holger),$ \\
						$(Gertrud,Holger), (Katja, Nina) \} \bigcup \{${dazu sym. Tupel}$\}$ \pause
			\item dann ist $R^0=\{ (Martin,Martin), ..., (Holger,Holger) \}$
			\item und $R^1=R$ und \pause
			\item $R^2=\{ (Martin,Nina), (Martin,Gertrud), (Martin,Martin),$ \\
						$(Lars,Nina), (Lars,Lars), (Nina,Gertrud),(Nina,Martin),$ \\
						$(Nina,Nina), (Nina,Lars), (Katja,Katja), (Katja,Holger), $ \\
						$(Gertrud,Gertrud), (Gertrud,Martin), (Gertrud,Nina), $ \\
						$(Holger,Holger), (Holger,Katja)\}$
			\item $R^*=$ ? \pause Ist $R^*$ eine �quivalenzrelation?
			%\item wegen Symmetrie und Unsinnigkeit der Reflexivit�t, entf�llt Einiges % Funktionen sind kommutativ, Relationen symmetrisch
		\end{itemize}
	\end{block}
\end{frame}


\subsection*{}
\begin{frame}
	\frametitle{Relationen graphisch}
	\begin{block}{Ihr seid dran...}
		\begin{enumerate}
			\item �berlegt euch, wie eine Relation graphisch aussehen k�nnte. Zeigt ein Beispiel mit mindestens $4$ verschiedenen Elementen %Grundmenge?
			\item Wie sieht nun graphisch die reflexiv-transitive H�lle aus?
		\end{enumerate}
	\end{block} \pause
    \begin{block}{m�gliche Darstellung}
    	\begin{itemize}
          \item Relation als Pfeile von Element zu Element
          \item Relation als Matrix, d.h. wenn xRy ist Feld [x,y] $== 1$
        \end{itemize}
    \end{block}
\end{frame}

\section{Abschluss}
% Studis anzuregen dar�ber nachzudenken, ob sie wirklich alles wissen, ansonsten nachlesen oder fragen nachtr�glich stellen, dann kann in der n�chsten Woche nochmal drauf eingegangen werden
\subsection*{}
\begin{frame}
	\frametitle{Zum Schluss...}
	\begin{block}{Was ihr nun wissen solltet!}
	\begin{itemize}
		\visible<2->{\item Was sind Grammatiken?}
		\visible<3->{\item Was l�sst sich aus ihnen ableiten?}
	\end{itemize}
	\end{block}

	\visible<4->{
	\begin{block}{Ihr wisst was nicht?}
		Stellt \textbf{jetzt} Fragen!
	\end{block}}
\end{frame}

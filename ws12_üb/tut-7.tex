\input{../skript/gbi-macros.tex}
\documentclass[12pt]{article}

\usepackage{german}
\usepackage{mdwlist}
\usepackage{enumerate}
\usepackage{color}
\usepackage[utf8]{inputenc}
\usepackage{graphicx}
\usepackage{amsmath}
\usepackage{amssymb}
\usepackage{wasysym}
\usepackage[paper=a4paper]{geometry}
\usepackage{graphicx}
\usepackage{pst-all}
\usepackage{booktabs}
\usepackage{xspace}

\usepackage[amsmath,thmmarks]{ntheorem}

\theoremheaderfont{\bfseries}
\theoremstyle{margin}
\usepackage{amsmath}
\usepackage{checkend}
\usepackage{color}
\usepackage{graphicx}
\usepackage{picinpar}
\usepackage[alsoload=binary]{siunitx}
\usepackage{tikz}
\usetikzlibrary{arrows}
   \usetikzlibrary{automata}
   \usetikzlibrary{matrix}
\usepackage{url}

\usepackage[amsmath,thmmarks]{ntheorem}

\usepackage[blue]{thwregex}

\def\tightlist{\setlength{\itemsep}{0pt}\setlength{\parsep}{0pt}\setlength{\parskip}{0pt}}
\renewcommand{\dh}{d.\,h.\@\xspace}

\theoremheaderfont{\bfseries}
\theoremstyle{margin}

\newcommand{\thwtheoremindent}{\relax}

\newtheorem{aaaaaa}{aaaaaa}[section]
% \theoremsymbol{\ensuremath{\Diamond}}
\theorembodyfont{\upshape} \newtheorem{definition}[aaaaaa]{\thwtheoremindent Definition.}

\theorembodyfont{\upshape} \newtheorem{lemma}[aaaaaa]{\thwtheoremindent Lemma.}
\newenvironment{beweis}%
  {\begin{proofinternal}}%
  {\endofproof\end{proofinternal}}%
\newdimen\endofproofsize\endofproofsize=0.5em
\def\endofproof{~\quad\hglue\hsize minus\hsize
                  \hbox{\vrule height \endofproofsize width \endofproofsize}\par}


\begin {document}


\bibliography{../skript/gbi}

\setcounter{section}{6}

\section{Graphen}
\subsection{Gerichtete Graphen und Teilgraphen}

\subsubsection{Graphen:}
  \begin{itemize}
  \item zur Motivation:
    \begin{itemize}
    \item Einbahnstraßensystem, .....
    \item wie würde man Zweibahnstraßen modellieren? die Fahrspuren
      für beide Richtungen separat
    \item \textbf{Achtung:} analoge Idee für Autobahmodellierung
      (mehrere Spuren in die gleiche Richtung zwischen zwei Knoten)
      geht nicht: $E\subseteq V\x V$ erlaubt nur: es gibt \emph{keine}
      Kante von $x$ nach $y$ oder es gibt \emph{eine}
      Kante. Sogenannte Mehrfachkanten sind bei uns nicht möglich und 
     geben bei Lösungen auf jeden Fall Punktabzug.
    \end{itemize}
  \item Beispiele malen:
    \begin{itemize}
    \item einschließlich Extremfälle mit $0$ Kanten \bzw maximal vielen
      Kanten; mit Schlingen und ohne Schlingen.
    \item Sonderfälle wie Bäume (siehe weiter unten) und Zyklen
    \item Beim Malen darauf hinweisen, dass man den gleichen Graphen
      unterschiedlich hinmalen kann, \zB den vollständigen Graphen $K_4$ mit sich kreuzenden
      Kanten oder ohne.
    \end{itemize}
  \item Eigenschaften von Graphen an Beispielen diskutieren
    \begin{itemize}
    \item beim Straßensystem: Man möchte von jedem Knoten zu jedem
      kommen.
    \item Wenn die Knoten Rechner sind und die Kanten Kabel: Man
      möchte von $x$ nach $y$ nur über möglichst "`wenige"'
      Kanten laufen müssen (egal wo $x$ und $y$)
    \end{itemize}
  \item Wenn ein Graph $n$ Knoten hat:
    \begin{itemize}
    \item Wieviele Kanten kann er maximal haben, wenn Schlingen
      erlaubt sind? $n^2$
    \item Wieviele Kanten kann er maximal haben, wenn er schlingenfrei
      ist? $n(n-1)$ \\
    \end{itemize}
  \end{itemize}

\subsubsection{Definition Teilgraph:}
  \begin{itemize}
  \item Beachte: zu jeder Kante, die man in $E'$ haben will, müssen
    auch Anfangs- und Endknoten in $V'$ vorhanden sein!
  \item hinreichend großes Beispiel machen, bei dem sowohl
    $(\{0,1,2\},$ $\{(0,1),(0,2)\})$ als auch
    $(\{3,4,5\},\{(3,4),(3,5))\})$ Teilgraph ist:
    \begin{itemize}
    \item \textbf{Achtung:} formal sind das verschiedene (Teil-)Graphen
    \item \textbf{aha:} aber sie sehen gleich aus: so was nennt man
      isomorphe Graphen
    \end{itemize}
  \end{itemize}

\subsection{Pfade und Erreichbarkeit}

\subsubsection{Definition Pfade:}
  \begin{itemize}
  \item Beispiel machen, in dem zwar ein Pfad von $x$ nach $y$
    existiert, aber nicht umgekehrt.
  \item beachte: für aufeinanderfolgende Knoten im Pfad muss die Kante
    in die richtige Richtung weisen!
  \item Beachte: Knoten dürfen in Pfad mehrfach vorkommen
  \item Beispiel machen, in dem von $x$ nach $y$ unterschiedlich lange
    Pfade vorkommen.
  \end{itemize}

\subsection{Isomorphie von Graphen}
 \begin{itemize}
  \item Da hatten wir z.B. im WS10/11 eine Aufgabe auf dem Übungsblatt 8 zum Erkennen von isomorphen Graphen. Das macht das ganze vielleicht klar:
   \begin{itemize}
  \item Für welche der folgenden sechs Graphen gibt es einen
    Isomorphismus zu einem der anderen fünf Graphen? Geben Sie jeweils
    den zugehörigen Isomorphismus an.

    \psset{xunit=0.55cm, yunit=0.55cm, runit=0.55cm}
    \begin {pspicture}(0, 0)(26, 5) \cnodeput(2, 4){A}{$0$}
      \cnodeput(3.5, 3){B}{$1$} \cnodeput(2.9, 1){C}{$2$}
      \cnodeput(1.1, 1){D}{$3$} \cnodeput(0.5, 3){E}{$4$}
      \rput(2,0){$G_0$}

      \ncline{-}{A}{B}
      \ncline{-}{B}{C}
      \ncline{-}{E}{A}
      \ncline{-}{D}{E}
      \ncline{-}{C}{D}
      \ncline{-}{B}{E}
      
      \cnodeput(6.5, 4){A1}{$0$}
      \cnodeput(8, 3){B1}{$1$}
      \cnodeput(7.4, 1){C1}{$2$}
      \cnodeput(5.6, 1){D1}{$3$}
      \cnodeput(5, 3){E1}{$4$}
      \rput(6.5,0){$G_1$}  
      
      \ncline{-}{A1}{B1}
      \ncline{-}{B1}{C1}
      \ncline{-}{D1}{E1}
      \ncline{-}{E1}{A1}
      \ncline{-}{B1}{D1}
      \ncline{-}{E1}{C1}

      \cnodeput(11, 4){A2}{$0$}
      \cnodeput(12.5, 3){B2}{$1$}
      \cnodeput(11.9, 1){C2}{$2$}
      \cnodeput(10.1, 1){D2}{$3$}
      \cnodeput(9.5, 3){E2}{$4$}
      \rput(11,0){$G_2$}
      
      \ncline{-}{A2}{B2}
      \ncline{-}{C2}{D2}
      \ncline{-}{D2}{E2}
      \ncline{-}{E2}{A2}
      \ncline{-}{C2}{E2}
      \ncline{-}{D2}{B2}

      \cnodeput(15.5, 4){A3}{$0$}
      \cnodeput(17, 3){B3}{$1$}
      \cnodeput(16.4, 1){C3}{$2$}
      \cnodeput(14.6, 1){D3}{$3$}
      \cnodeput(14, 3){E3}{$4$}
      \rput(15.5,0){$G_3$} 
      
      \ncline{-}{A3}{B3}
      \ncline{-}{A3}{C3}
      \ncline{-}{A3}{D3}
      \ncline{-}{E3}{A3}
      \ncline{-}{C3}{E3}
      \ncline{-}{D3}{B3}

      \cnodeput(20, 4){A4}{$0$}
      \cnodeput(21.5, 3){B4}{$1$}
      \cnodeput(20.9, 1){C4}{$2$}
      \cnodeput(19.1, 1){D4}{$3$}
      \cnodeput(18.5, 3){E4}{$4$}
      \rput(20,0){$G_4$}
      
      \ncline{-}{E4}{B4}
      \ncline{-}{E4}{C4}
      \ncline{-}{E4}{D4}
      \ncline{-}{E4}{A4}
      \ncline{-}{C4}{D4}
      \ncline{-}{A4}{B4}

      \cnodeput(24.5, 4){A5}{$0$}
      \cnodeput(26, 3){B5}{$1$}
      \cnodeput(25.4, 1){C5}{$2$}
      \cnodeput(23.6, 1){D5}{$3$}
      \cnodeput(23, 3){E5}{$4$}
      \rput(24.5,0){$G_5$} 
      
      \ncline{-}{A5}{D5}
      \ncline{-}{A5}{C5}
      \ncline{-}{A5}{D5}
      \ncline{-}{C5}{B5}
      \ncline{-}{C5}{E5}
      \ncline{-}{D5}{B5}
      \ncline{-}{D5}{E5}
    \end{pspicture}
 
   \end{itemize}
 \end{itemize}

\subsection{Ein Blick zur\"uck auf Relationen}

\subsubsection{Pfade, $E^*$}
  \begin{itemize}
  \item $E^2$ ist wieder Relation auf $V$: kann man also als Graph
    malen: Beispiel machen
  \item analog für $E^3$, \dots
  \item und $E^*$ ist auch wieder eine Relation auf $V$: kann man also
    als Graph malen: Beispiel: aus Zyklus der Länge $5$ wird der
    sogenannte vollständige Graph $K_5$
  \end{itemize}
\subsection{Ungerichtete Graphen}

  \begin{itemize}
  \item \textbf{Achtung:} man reite noch mal auf der Formalisierung
    von Kanten herum:
    \begin{itemize}
    \item für $x\not=y$ ist $\{x,y\}$ eine zweielementige Menge,
      \emph{ohne} eine Festlegung von Reihenfolge
    \item für $x=y$ ist die Menge $\{x,y\}=\{x\}$ eine
      \emph{ein}elementige Menge
    \end{itemize}
  \item Wie ist das mit der Anzahl Kanten eines ungerichteten Graphen
    mit $n$ Knoten:
    \begin{itemize}
    \item Wieviele Kanten kann er maximal haben, wenn er schlingenfrei
      ist? $n(n-1)/2$\\
      Begründung: von jedem Knoten zu jedem
      \emph{anderen}; durch zwei, weil sonst jede Kante zweimal gezählt wird.
    \item Wieviele Kanten kann er maximal haben, wenn er Schlingen
      haben darf? $n(n+1)/2$ \\
      Begründung: $n+n(n-1)/2 = n(n+1)/2$
    \end{itemize}
  \end{itemize}

\subsection{Anmerkung zu Relationen}

\subsubsection{Äquivalenzrelationen:}

  Falls schon Fragen kommen: mit dem Bild einer
  Nicht"=Äquivalenzrelation anfangen und so lange Pfeile dazu malen,
  bis alle Forderungen erfüllt sind:
  \begin{itemize}
  \item Schlingen an allen Knoten
  \item zu jedem Pfeil hin auch der zurück
  \item wenn ein Pfad von $x$ nach $y$ existiert, dann auch eine
    direkte Kante
  \end{itemize}
  Ergebnis: einige Klumpen, äh, Cliquen (die den Äquivalenzklassen
  entsprechen)

\subsection{Graphen mit Knoten- oder Kantenmarkierungen}

\subsubsection{kantenmarkierte Graphen:}
  \begin{itemize}
  \item Noch mal einen Huffman-Baum hinmalen und diskutieren
  \item für Zahlen als Kantenmarkierungen siehe gleich
  \end{itemize}

\subsubsection{Graphen mit gewichteten Kanten}
  \begin{itemize}
  \item Beispielgraphen hinmalen und die Studenten kurze und lange
    Wege suchen lassen
  \end{itemize}

  \begin{itemize}
  \item Beispielgraphen hinmalen und die Studenten große Flüsse suchen
    lassen.
  \end{itemize}

\subsection{alte (Klausur-)Aufgaben}
 \begin{itemize}
  \item Aufgabe aus ÜB7 (WS08/09): Gegeben sei der Graph $G=(V, E)$ mit $V=\{\#0, \#1\}^3$ und $E=\{(xw,
  wy) \mid x, y \in \{\#0, \#1\} \land w \in \{\#0, \#1\}^2\}$.

   \begin{enumerate}[a)]
    \item Graphen zeichnen lassen.
    \item Geben Sie einen Zyklus in $G$ an, der außer dem Anfangs- und
      Endknoten jeden Knoten von $G$ genau einmal enthält.
    \item Geben Sie einen geschlossenen Pfad in $G$ an, der jede Kante von $G$
      genau einmal enthält.
  \end{enumerate}
 \item So ziemlich in jeder der letztjährigen Klausuren kam was zu Graphen dran z.B. http://gbi.ira.uka.de/archiv/2010/k-mar11.pdf \\
  oder http://gbi.ira.uka.de/archiv/2010/k-sep11.pdf. 

 Adjazenz-/Wegematrizen sind allerdings noch nicht bekannt.
 \end{itemize}
\end{document}

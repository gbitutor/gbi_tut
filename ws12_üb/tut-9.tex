\input{./gbi-macros.tex}
\documentclass[12pt]{article}

\usepackage{german}
\usepackage{mdwlist}
\usepackage{enumerate}
\usepackage{color}
\usepackage[utf8]{inputenc}
\usepackage{graphicx}
\usepackage{amsmath}
\usepackage{amssymb}
\usepackage{wasysym}
\usepackage[paper=a4paper]{geometry}
\usepackage{graphicx}
\usepackage{booktabs}

\usepackage[amsmath,thmmarks]{ntheorem}

\theoremheaderfont{\bfseries}
\theoremstyle{margin}
\usepackage{amsmath}
\usepackage{checkend}
\usepackage{color}
\usepackage{graphicx}
\usepackage{picinpar}
\usepackage[alsoload=binary]{siunitx}
\usepackage{tikz}
\usetikzlibrary{arrows}
   \usetikzlibrary{automata}
   \usetikzlibrary{matrix}
\usepackage{url}
\usepackage{xspace}

\usepackage{kbordermatrix}

\usepackage[amsmath,thmmarks]{ntheorem}

\usepackage[blue]{thwregex}
\usepackage{thwtextabbrevs}

\def\tightlist{\setlength{\itemsep}{0pt}\setlength{\parsep}{0pt}\setlength{\parskip}{0pt}}
\renewcommand{\dh}{d.\,h.\@\xspace}

\theoremheaderfont{\bfseries}
\theoremstyle{margin}

\newcommand{\thwtheoremindent}{\relax}

\newtheorem{aaaaaa}{aaaaaa}[section]
\theorembodyfont{\upshape} \newtheorem{definition}[aaaaaa]{\thwtheoremindent Definition.}

\theorembodyfont{\upshape} \newtheorem{lemma}[aaaaaa]{\thwtheoremindent Lemma.}
\newenvironment{beweis}%
  {\begin{proofinternal}}%
  {\endofproof\end{proofinternal}}%
\newdimen\endofproofsize\endofproofsize=0.5em
\def\endofproof{~\quad\hglue\hsize minus\hsize
                  \hbox{\vrule height \endofproofsize width \endofproofsize}\par}

\def\Red#1{\textcolor{darkred}{#1}}

\begin {document}


\bibliography{../skript/gbi}

\setcounter{section}{8}

\section{Quantitative Aspekte von Algorithmen}

\subsection{Gro\ss-O-Notation}
\subsubsection{Ignorieren konstanter Faktoren}

\begin{itemize}
 \item   Wir machen das ein bisschen anders als andere:
  \begin{itemize}
  \item Erst wird $\Theta$ eingef\"uhrt, und danach $O$:
    \begin{itemize}
    \item ich finde, dass $\Theta$ das n\"aher liegende ist, und man
      kann sich erst mal drauf beschr\"anken, dass Ignorieren konstanter
      Faktoren kennenzulernen
    \item die Verallgemeinerungen zu $O$ und $\Omega$ sind evtl dann
      leichter
    \end{itemize}
  \item Wir f\"uhren erst eine \"Aquivalenzrelation $\asymp$ ein, und dann
    $\Theta(f)$ als \"Aquivalenzklasse (ohne dieses Wort schon zu
    benutzen) von Funktionen.
  \item Auch so ist hinterher leicht zu sehen, dass $\Theta(f)=O(f)\cap
    \Omega(f)$.
  \item Achtung: einiges k\"onnte man auch leicht unter Verwendung von
    $\lim$, oder genauer mit $\limsup$ argumentieren, aber die Inwis
    haben vermutlich noch keine Grenzwerte.
  \end{itemize}

\item
  $\Theta$ und Polynome: Man versuche klar zu machen, dass immer
  $f\asymp g$ ist, wenn $f$ und $g$ Polynome gleichen Grades sind,
  also \zB $42n^6-33n^3+222n^2 -15 \asymp 66n^6+55555n^5$. Das kann
  man \zB in Anlehnung an $n^3+5n^2\asymp 3n^3-n$ aus der Vorlesung
  machen.

\item  Beispiel: Logarithmenfunktionen haben alle gr\"o\ss enordnungsm\"a\ss ig das
  gleiche Wachstum:
  \begin{itemize}
  \item Logarithmen sind ja wohl definitiv Schulwissen. Trotzdem
    darauf vorbereitet sein, dass Fragen kommen. Also: F\"ur $a\in\Rplus$
    und $n\in\N_+$ ist $\log_a(n)$ die Zahl mit $a^{\log_a(n)} = n$.
    Beachte: $n\geq1$, da $\log 0$ nicht definiert.
  \item Man zeige: $\log_2(n) \in\Th{\log_8(n)}$ \\
    \begin{itemize}
    \item man beginne vielleicht mit Beispielen:

      \begin{tabular}{*{6}{>{$}r<{$}}}
        n         & 1 & 8 & 64 & 512 & 4096 \\
        \log_8(n) & 0 & 1 &  2 &   3 &    4 \\
        \log_2(n) & 0 & 3 &  6 &   9 &   12 \\
      \end{tabular}

      da sollte man doch was sehen \dots
    \item dann rechnen: $n = 8^{\log_8 n} = (2^3)^{\log_8(n)} = 2
      ^{3\log_8(n)}$, also gilt f\"ur alle $n\geq 1$: $\log_2(n) = 3
      \log_8(n)$ und $\log_8(n)=\frac{1}{3}\log_2(n)$
    \item wenn das klar ist, dann wohl auch \dots
    \end{itemize}
  \item allgemein: $\log_b(n) \in\Th{\log_a(n)}$, denn
    \[
    b^{\log_b(n)} = n = a^{\log_a(n)} = (b^{\log_b(a)})^{\log_a(n)} = b^{\log_b(a) \cdot \log_a(n) }
    \]
    also liefert (Exponentiation ist injektiv) der Vergleich der
    Exponenten (oder anders gesagt: Logarithmieren beider Seiten):
    $\log_b(n) = \log_b(a) \cdot \log_a(n)$ \\
    also f\"ur $c'=c=\log_b(a)$ und alle $n\geq 1$ gilt: $c \log_a(n)
    \leq\log_b(n) \leq c' \log_a(n) $
  \item Man kann also einfach $\Th{\log n}$ schreiben, ohne die Basis
    anzugeben, denn sie ist egal.
  \item Und mir f\"allt auf, dass eine Aufgabe auf dem aktuellen \"Ubungsblatt dem ganzen doch recht \"ahnelt. Wenn die Leute im Tut also 
  keine Probleme mit $\log$ haben, muss man das ganze auch nicht so im Detail besprechen.
  \end{itemize}

\end{itemize}

\subsubsection{Notation f\"ur obere und untere Schranken des Wachstums}

\begin{itemize}
 \item zum Thema $\Oh{}$:
  \begin{itemize}
  \item Damit die Studenten ein besseres Gef\"uhl f\"ur $\Oh{\cdot}$
    bekommen, bitte noch mal genau $n^a\in \Oh{n^b}$ falls $a\leq b$
    betrachten.
  \item Aber damit da kein falscher Eindruck entsteht: \textbf{Bitte
      beachten:} $\preceq$ und $\succeq$ sind \emph{keine} totalen
    Ordnungen. Es gibt unvergleichbare Funktionen. \ZB
    \begin{align*}
      f(n) &=
      \begin{cases}
        1 & \text{ falls $n$ gerade} \\
        n & \text{ falls $n$ ungerade} \\
      \end{cases} \\
      g(n) &=
      \begin{cases}
        n & \text{ falls $n$ gerade} \\
        1 & \text{ falls $n$ ungerade} \\
      \end{cases} \\
    \end{align*}
    Es gilt \emph{nicht} $g\preceq f$, es gilt \emph{nicht} $f\preceq
    g$ und es gilt erst recht \emph{nicht} $f\asymp g$.

    Und das liegt auch nicht daran, dass die Funktionen so hin und her
    springen; f\"ur monoton wachsende Funktionen kann man so etwas auch
    machen; so etwas war z.B. auf \"UB9 im Jahre 2008 dran
  \end{itemize}

\item  Zu $\Om{\cdot}$: vielleicht auch ein paar einfache Beispiele: Macht
  es den Studenten Probleme, sich von $n^2\in\Om{\log n}$ zu
  \"uberzeugen?
\end{itemize}

\subsubsection{Die furchtbare Schreibweise}

  \begin{itemize}
  \item Folgendes ist sehr unsch\"one Variante der O-Notation, aber weit verbreitet: \\
  Man schreibt
    % 
    \begin{align*}
      g(n) = \Oh{f(n)}  &\text{\quad statt\quad }  g(n) \in \Oh{f(n)} \;, \\
      g(n) = \Th{f(n)}  &\text{\quad statt\quad }  g(n) \in \Th{f(n)} \;, \\
      g(n) = \Om{f(n)}  &\text{\quad statt\quad }  g(n) \in \Om{f(n)} \;.
    \end{align*}
  \item Ausdr\"ucke auf der linken Seite \Red{sind keine Gleichungen!}
  \item Daher bitte immer \Red{gro\ss e Vorsicht} walten lassen:
    \begin{itemize}
    \item Es ist \Red{falsch}, aus $g(n)=\Oh{f_1(n)}$ und
      $g(n)=\Oh{f_2(n)}$ zu folgern, dass $\Oh{f_1(n)}=\Oh{f_2(n)}$ ist.
    \item Es ist \Red{falsch}, aus $g_1(n)=\Oh{f(n)}$ und
      $g_2(n)=\Oh{f(n)}$ zu folgern, dass $g_1(n)=g_2(n)$ ist.
    \end{itemize}
  \item Bitte Fragen beantworten. ABER: Ich sehe zwar einen Grund so etwas
  lesen zu k\"onnen, aber keinen Grund diesen Unfug schreibenderweise zu
  \"uben.
  \end{itemize}

\subsection{Matrixmultiplikation}
\subsubsection{R\"uckblick auf die Schulmethode}


  \begin{itemize}
  \item Wird eigentlich in der Vorlesung nochmal besprochen
  \item Trotzdem: vielleicht muss man das noch ein bisschen
    erkl\"aren. Man nehme einfach $4\x 4$-Matrizen und sehe sich \zB
    $c_{11}=a_{11}b_{11} + a_{12}b_{21} + a_{13}b_{31} +
    a_{14}b_{41}$ an:

    Man schreibe sich einige der Bl\"ocke $A_{11}$ \usw hin. Dann sieht
    man: Der erste Teil $a_{11}b_{11} + a_{12}b_{21}$ "`kommt
    von"'/"`passt zu"' $A_{11}B_{11}$ und der zweite Teil
    $a_{13}b_{31} + a_{14}b_{41}$ "`kommt von"'/"`passt zu"'
    $A_{12}B_{21}$.
  \end{itemize}

\subsubsection{Algorithmus von Strassen}

\begin{itemize}
 \item Dient in erster Linie als Beispiel f\"ur eine effektivere Matrixmultiplikation. F\"ur das Tutorium vielleicht eher uninteressant.
 \item 
  Weitere \"Ubungsm\"oglichkeit: Codeschnipsel aus Sneltings Folien von
  2008 f\"ur Berechnung der Binomialkoeffizienten:
\begin{verbatim}
  static int binom(int n, int k) {
    assert n >= k && k >= 0;
    if (k == 0 || k == n) {
      return 1;
    } else {
      return binom(n - 1, k - 1) + binom(n - 1, k);
    }
  }
\end{verbatim}
  Diskussion: Wieviele Aufrufe von \verb|binom| in Abh\"angigkeit von
  $n$ werden bei der Berechnung eines $\binom{n}{k}$ gemacht? Im
  Detail ist das nicht ganz sch\"on zu machen. Man \"uberzeuge sich aber
  (mit Hilfe eines Beispiels?) davon, dass man mindestens $2^k$ Aufrufe
  der Form $binom((n-k),x)$ mit $0\leq x\leq k$ hat. Das sind im Fall
  $k=n/2$ also immerhin $\left(\sqrt{2}\right)^n$.

\end{itemize}

\subsection{Asymptotisches Verhalten "`implizit"' definierter Funktionen}
Der Stoff ab hier kommt erst n\"achste Woche dran.
\begin{itemize}
 \item 
  Zum Mastertheorem kommen wir erst am 19.12.; dann sollte
  in Programmieren rekursives Suchen/Sortieren dran gewesen sein. Das
  gibt dann noch mal Motivation f\"ur
  \[
  T(n) = a T\left(\frac{n}{b}\right) + f(n)
  \]

\item  Mastertheorem
  \begin{itemize}
  \item Fall 2: $f(n) \in \Th{n^{\log_b a}}$ schl\"agt bei Quicksort zu
    \begin{itemize}
    \item Formel anwenden liefert $n\log n$
    \end{itemize}
  \item Fall 3: nur bei Nachfragen diskutieren \dots
  \item statt dessen darauf hinweisen, dass einem das Mastertheorem
    nicht weiterhilft, wenn man eine Probleminstanz anders zerhackt,
    wie etwa bei  $(n+1)!=(n+1)*n!$ oder
    \[
    \binom{n}{k}=\begin{cases}
      0 & \text{ falls } k=0\lor k=n\\
      \binom{n-1}{k-1}+\binom{n-1}{k} & \text{ sonst}
    \end{cases}
    \]
   % (siehe weiter vorne)
  \end{itemize}

\end{itemize}

\end{document}

\section[Einstieg]{Guten Morgen...}
\subsection*{}
\begin{frame}{Zum Warmwerden...}
  Die vollständige Induktion...
    \begin{enumerate}
    \item { \only<2->{ \color{red} }
    ... besteht aus Induktionsanfang und Induktionsschritt.
    }
    \item { \only<2->{ \color{green!50!black} }
    ... wird zum beweisen von Aussagen genutzt, die sich auf ein beliebiges Element ($n$) einer Rekursion, Formel, etc. beziehen.
    }
    \item { \only<2->{ \color{red} }
    ... beginnt immer mit dem Nachweis für $n=0$.
    }
    \end{enumerate}

  Für zwei Mengen $M_1$ und $M_2$ gilt...
    \begin{enumerate}
    \item { \only<3->{ \color{green!50!black} }
    ... sind gleich, wenn: $M_1 \subseteq M_2$ und $M_2 \subseteq M_1$
    }
    \item { \only<3->{ \color{green!50!black} }
    ... $\exists$ bijektive Abbildung von $M_1$ nach $M_2$, wenn $\left|M_1\right| = \left|M_2\right|$
    }
    \end{enumerate}
\end{frame}

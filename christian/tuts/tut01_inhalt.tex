%für ÜB2: \item Mengen

% Abgabe jeweils Freitag 12.30, hier im Untergeschoss bei den Briefkästen
% -direkt danach werden die Kästen geleert, wer danach abgibt bekommt keine Punkte!


% betrachte tut01 aus 08,9,10, welches passt?


% bei nächstem Tutorentreffen Tutoren ansprechen, die sofort korrigieren!


% ÜB1
% 1.4b): einzig erlaubte Verknüpfung ist "Herz", d.h. nicht erlaubt ist "nicht", erlaubt ist "(", ")"
% Beispiele zu injektiv  (jedem rechten nur ein "eindeutiges" linkes zugeordnet), surjektiv
% Beispiele zu linkstotal, rechtseindeutig (jedem linken nur ein "eindeutiges" rechtes zugeordnet)


% 1.1 wahrheitstabelle beispiele, ausdruck vereinfachen

% 1.5: gab ein update, "alle können sehen" war gemeint


% was ist
 % 1000001 ?
% wie kam Professorin auf 65?
% wie auf "730"?
% wie auf A?


% ----- ab hier eigentlicher Inhalt -------------------------------------------
\section[Organisation]{Organisatorisches}
%\subsection*{}
\subsection*{}
\begin{frame}
\frametitle{Euer Tutor}
        \begin{block}{"Uber mich...}
                \begin{description}
                        \item[Name:] Christian Jülg
                        \item[Alter:] 28 Jahre
                        \item[Studium:] Informatik im 17. Semester
                \end{description}
        \end{block}

        \vspace{2ex}

        \begin{block}{Kontakt}
                \begin{description}
                        \item[] gbi-tutor@gmx.de
                        \item[] \url{http://gbi-tutor.blogspot.com}
                        \item[] \url{http://twitter.com/gbitutor}
                \end{description}
        \end{block}
\end{frame}

%Nochmal nachschauen wer wann Tut hat
\subsection*{}
\begin{frame}
        \frametitle{Andere Tutorien}
		% \begin{block}{Die Tutorien}
		% Die Folien gefallen euch? Hier gibts mehr davon:
		% \begin{description}
			% \item[Montag:] Martin Schadow
		% \end{description}
		% \end{block}
		% \only<2>{\vspace{2ex}}
		% \only<2>{
        \begin{block}{Hinweis}
              \begin{itemize}
                \item ihr könnt beliebige Tutorien besuchen\\ (solange genug Platz ist)
                \item die Blätter werden aber vom festgelegten Tutor korrigiert und zurückgegeben
                \item ein kompletter Tutorienwechsel ist nur möglich, wenn im anderen Tutorium Platz ist
              \end{itemize}
        \end{block}
        %}
\end{frame}


\subsection*{}
\begin{frame}
\frametitle{Ihr seid dran...}
	\begin{block}{"Uber euch...}
		Stellt euch alle bitte kurz vor... 
		\begin{description}
			\item[Vorname:]Wie hei"st ihr?
			\item[Studiengang:] Was studiert ihr?
		\end{description}
	\end{block}
  \pause
  	\begin{block}{Inhalt des Tutoriums}
		Was erwartet ihr vor allem von diesem Tutorium?
		\begin{itemize}\pause
			\item Besprechung der alten Blätter?\pause
			\item Wiederholung der Vorlesung?\pause
			\item Vorbereitung auf nächstes Blatt?
		\end{itemize}
	\end{block}
\end{frame}

\subsection*{}
\begin{frame}
        \frametitle{Grundbegriffe der Informatik}
        \begin{block}{zur Veranstaltung}
                \begin{description}
                \item[Homepage:] \url{http://gbi.ira.uka.de}
                \item[Skript, Folien:] zum Download auf der Homepage
                \item[Schein:] Voraussetzung sind 50\% der Punkte aus den Aufgabenbl"attern\\  Teil des Moduls (für Infos, InWis und Physiker)
                \end{description}
        \end{block}

        \vspace{2ex}

        \begin{block}{Klausurtermine}
                \begin{description}
                  \item[Klausur:] 7. März 2013
                  \item[Nachklausur:] findet idR zum Ende des Sommersemesters statt\\
                          (Teilnahme auch als Erstversuch möglich)
                \end{description}
        \end{block}
\end{frame}

\subsection*{}
\begin{frame}
        \frametitle{Und nun zum Tutorium...}
        \only<1>{
        \begin{block}{Tutorium Nr. 21}
                \begin{itemize}
                        \item Dienstag, 15:45 Uhr - 17:15 Uhr im SR -119 Infobau
                        \item Folien werden nach dem Tutorium auf der Website ver"offentlicht
                        \item ein Tutorium ist \textbf{keine} Vorlesung
                        \item aktive Mitarbeit wird erw"unscht und erwartet 
                        \item	keine Scheu vor falschen Antworten!
                        \item je mehr falsche Antworten, desto mehr Lerneffekt für die Gruppe ;)
                \end{itemize}
        \end{block}}
 \end{frame}

\subsection*{}
\begin{frame}
        \frametitle{"Ubungsbl"atter}
        \begin{block}{"Ubungsbl"atter}
                \begin{itemize}
                		\only<1>{\item Die Aufgaben handschriftlich bearbeiten
                		\item Gruppenarbeit m"oglich und empfohlen, allerdings muss jeder seine eigenen L"osungsbl"atter abgeben 
                        \item alle Bl"atter getackert (1x!) + spezielles Deckblatt (beim "Ubungsblatt mit dabei)
                        \item Deckblatt zwingend erforderlich
                        \item offensichtlich abgeschriebene L"osung: 0 Punkte}
                \end{itemize}
        \end{block}
\end{frame}


\section[Don't Panic!]{Don't Panic!}
\subsection*{}
\begin{frame}
	\frametitle{Don't Panic!}
	\only<1>{
		\begin{block}{Wovon wir ausgehen...}
		\begin{itemize}
			\item Fachhochschul- oder Hochschulreife
			\item halbwegs logisches Denken
			\item keine Vorkenntnisse in der Informatik
			\item Beherrschung der deutschen Sprache
			\item logisch gegliederte Ausdrucksweise und die Fähigkeit ganze Sätze zu formulieren
		\end{itemize}
	\end{block}
	} \only<2>{
	\begin{block}{Was wir erwarten...}
		\begin{itemize}
			\item Lernbereitschaft
			\item eigenständiges Erarbeiten von Lösungen
			\item Lösungen die dem Ziel der Aufgabe entsprechen
			\item das Einhalten der Rahmenbedingungen für die Abgabe der Aufgabenblätter
		\end{itemize}
	\end{block}
}
\end{frame}

\subsection*{}
\begin{frame}
	\frametitle{Easy durch GBI...}
	\begin{itemize}
		\item Bearbeitet die Übungsblätter! Das bringt euch nicht nur einen Schein,
		sondern auch Übung für die Klausur. (selbst falls ihr keinen Schein braucht)
		\pause
		\item Fangt früh mit dem Wiederholen für die Klausur an! Die Klausur ist wahrscheinlich wieder kurz nach Semesterende\pause
		\item Schiebt die Klausur nicht! Schreibt die Hauptklausur!
			(Denkt dran: Orientierungsprüfung)	\pause
		\item Lasst euch im Tutorium nicht nur berieseln, sondern arbeitet aktiv mit!
	\end{itemize}	\pause
 \begin{block}{Ohne Gewähr}
 	Alle Angaben wie immer ohne Gewähr und übertragbar auf andere Vorlesungen
 \end{block}
\end{frame}

\subsection*{}
\begin{frame}
\frametitle{Fragen über Fragen...}
\begin{block}{F1 - Hilfe}
	\begin{itemize}
  \item  Wenn ihr Fragen zur Vorlesung habt \\ \pause
          \ldots Skript lesen
  \item  Wenn ihr Fragen zum Übungsblatt habt \\ \pause
          \ldots Skript lesen
  \item  Wenn ihr das Skript lesen wollt \\ \pause
          \ldots Skript lesen ;) \pause
			\item ins Forum schauen und ggf. ein neues Thema zu eurem Problem aufmachen\pause
			\item Tutor fragen :)	\pause
			\item eine Mail an den Übungsleiter	\pause
			\item ggf. Dozenten nach der Vorlesung ansprechen (beißen in der Regel nicht oder nur bei großem Hunger)
	\end{itemize}
\end{block}
\end{frame}


\section[Blatt 1]{Aufgabenblatt 1}
\subsection*{}
\begin{frame}
        \frametitle{Aufgabenblatt 1}
        \begin{block}{Blatt 1}
        \begin{itemize}
        	\item Abgabe: 26.10.2012 12.30 Uhr im Keller des Infobaus
          \item wer nach Leerung der Kästen abgibt, bekommt keine Punkte!
        	\item Punkte: maximal 19
        \end{itemize}
        \end{block}
        \begin{block}{Themen}
        \begin{itemize}
			\item Mengen
        	\item Relationen
        	\item Abbildungen
        \end{itemize}
        \end{block}
\end{frame}


%\section[Mengen]{Mengenlehre}
\begin{frame}
	\frametitle{Mengenlehre}
	\begin{block}{Teilmengen}
		\begin{itemize}
			\item Was sind die Teilmengen von $\{11,22\}$?\only<2->{\\ $\{ \}, \{11\}, \{22\}, \{11,22\}$}
			\only<3->{\item Was sind die Teilmengen von $\{11,22,33\}$?}\only<4->{\\ $\{ \} , \{11\}, \{22\}, \{33\},\{11,22\},\{11,33\},\{22,33\},\{11,22,33\}$}
			\only<5->{\item Handelt es sich dabei um echte Teilmengen?} \only<6->{Nein! Die letzte nicht.}
		\end{itemize}	
	\end{block}
	
	\only<7->{
	\begin{block}{kartesisches Produkt}
		\begin{itemize}
			\item Das karthesische Produkt zweier Mengen ist definiert als $A \times B := \left\{(a, b)|a \in A, b \in B\right\}$ 
			\only<8->{ \item Berechne $\{1,2,3\} \times \{0,1\}$} \only<9->{ $=\{(1,0),(1,1),(2,0),(2,1),(3,0),(3,1)\}$ }.
		\end{itemize}
	\end{block}}
\end{frame}

\begin{frame}
	\frametitle{Relationen und Abbildungen}
	\only<1->{
	\begin{block}{Ein paar Definitionen...}
	 \begin{itemize}
		\item $R \subseteq A \times B$ hei"st {\em Relation}.
		\item Eine Relation  $R \subseteq A \times B$ hei"st
		\begin{itemize}
			\item linkstotal: wenn f"ur jedes $a \in A$ ein $b \in B$ existiert mit $(a,b) \in R$
			\item rechtseindeutig: wenn es f"ur kein $a \in A$ zwei $b_1 \in B$ und $b_2 \in B$ mit $b_1 \neq b_2$ gibt, sodass sowohl $(a,b_1) \in R$ und $(a,b_2) \in R$ ist
		\end{itemize}
\end{itemize}
\end{block}}
  \end{frame}

  
\subsection{Funktionen}
\begin{frame}
	\frametitle{Funktionen}
	\begin{block}{Ein paar Definitionen...}
	Wenn $f:A \rightarrow B$ und $g:B \rightarrow C$ totale Funktionen sind, ist
  \begin{itemize}
	  \item $A$ der \emph{Definitionsbereich} von $f$ \pause
	  \item $B$ der \emph{Zielbereich} von $f$ \pause
	  \item $f(A)$ heißt der \emph{Wertebereich} von $f$ \pause
	  \item Die Komposition der Funktionen $g \circ f$ formal definiert als $g\circ f: A \rightarrow C$ mit $(g\circ f)(a)=g(f(a))$
  \end{itemize}
  \end{block}
\end{frame}


\begin{frame}
  \begin{block}{Ihr seid dran...}
  	\begin{itemize}
 			\item Gegeben seien
 				\\ $f:\mathbb N_0 \rightarrow \mathbb N_0:x\mapsto x+1$
 			und \\ $g:\mathbb N_0 \rightarrow\mathbb N_0:x\mapsto x^2$
 				\begin{itemize}
 					\item Notiere $f \circ g$ \pause
 					 \only<2->{\\ $(f \circ g)(x) = f ( x^2) = x^2 +	1$}
 					\item Notiere $g \circ f$ \pause
 					 \only<3->{\\ $(g\circ f)(x) = g(f(x)) = g(x+1) = (x+1)^2$}
 				\end{itemize}
  	\end{itemize}
  \end{block}
\end{frame}

\begin{frame}
\frametitle{Relationen und Abbildungen}
  \begin{block}{Ihr seid dran...}
  	\begin{itemize}
 			\item Wann sind zwei Funktionen verschieden?
 				\pause
 					\begin{itemize}  \pause
 					 	\item Wenn für min. ein Argument verschiedene Funktionswerte
 					 	herauskommen \pause
   					 	\item wenn die Definitionsbereiche verschieden sind \pause
    					\item wenn die Zielbereiche verschieden sind: \pause
      					\begin{itemize}
      						\item bei uns sind
      							\\ $f_1:\mathbb N_0 \rightarrow \mathbb N_0: x\mapsto x+1$ und
      							\\ $f_2:\mathbb N_0 \rightarrow \mathbb N_+: x\mapsto x+1$
      							\\\emph{verschiedene} Funktionen.
  						\end{itemize}
 					\end{itemize}
  	\end{itemize}
  \end{block}
 \pause
\begin{block}{Abbildungen* hei"sen auch\ldots}
    *Abbildungen bzw. Funktionen sind rechtseindeutig und linkstotal
    \begin{itemize}
      \item injektiv, wenn sie \pause linkseindeutig sind
      \item surjektiv, wenn sie \pause rechtstotal sind
      \item bijektiv, wenn sie \pause injektiv und surjektiv sind
    \end{itemize}
\end{block}
\end{frame}


\begin{frame}
	\frametitle{Aussagenlogik I}
	 \begin{block}{Was ist das?}
	 H"aufig wollen wir Aussagen formalisieren und analysieren. Ein Mittel dazu ist die Aussagenlogik, die sich auf ganze Aussagen und ihren Verbindungen beschr"ankt.
	 \end{block}
	 
	 \begin{block}{Junktoren}
	 Folgende Junktoren helfen Aussagen zu verkn"upfen:
	 \begin{itemize}
	 	\item[$\neg$] Negation 
	 	\item[$\land$] Konjunktion
	 	\item[$\lor$] Disjunktion
	 	\item[$\rightarrow$] Implikation $A \rightarrow B \leftrightarrow \neg A \lor B$
	 \end{itemize}
	 \end{block}
\end{frame}

\begin{frame}
	\frametitle{Aussagenlogik II}
	\begin{block}{Ihr seid dran...}
	Formalisiert folgende Aussage:
	\\"`Wenn der Hahn kräht auf dem Mist, ändert sich das Wetter oder es bleibt wie	es ist"'
	\\Tipp: Führt Symbole für Teilaussagen ein!
	\\ \only<2->{
	\begin{itemize}
		\item $A$: "`Der Hahn kräht"'
		\item $B$: "`Das Wetter ändert sich"'
		\item \only<2>{$C$}\only<3->{$\neg B$}: "`Das Wetter ändert sich nicht"'
	\end{itemize}
\only<3->{Und zusammen: $A \rightarrow ( B \lor \neg B )$}

	\only <4->{Ist diese Aussage wahr? Lege eine Wahrheitstabelle an, um sie zu überprüfen!}
	}
	\end{block}
\end{frame}


\begin{frame}
	\frametitle{Aussagenlogik III}
	\begin{block}{Einige Gesetze}
	Folgende Gesetze helfen beim Umgang mit Aussagen:
	%Gesetze $(x, y, z \in A)$
	\begin{tabular}{lll}
		 Assoziativit"at 	& $(x \wedge y) \wedge z \equiv x \wedge (y \wedge z)$  &\\
						& $(x \vee y) \vee z \equiv x \vee (y \vee z)$ \\
		Kommutativit"at 	& $x \wedge y \equiv y \wedge x$ & $x \vee y \equiv y \vee x$ \\
		Idempontenz 		& $x \wedge x \equiv x$ & $x \vee x \equiv x$ \\
		Distributivit"at 	& \multicolumn{2}{l}{$x\wedge(y \vee z) \equiv (x \wedge y) \vee (x \wedge z)$ } \\
					& \multicolumn{2}{l}{$x\vee(y \wedge z) \equiv (x \vee y) \wedge (x \vee z)$} \\
	\end{tabular}
	\end{block}
	\end{frame}
  
  
\section[???]{Fragen???}
\begin{frame}
		\frametitle{Alle Klarheiten beseitigt?}
		\begin{center}
			\includegraphics[height=5cm]{comics/fragen.jpg}\\
		\end{center}
\end{frame}

\begin{frame}
	\frametitle{Wenn doch noch Fragen auftauchen...}
	        \begin{block}{Kontakt}
                \begin{description}
                        \item[] gbi-tutor@gmx.de
                        \item[] \url{http://gbi-tutor.blogspot.com}
                        \item[] \url{http://twitter.com/gbitutor}
                \end{description}
        \end{block}
\end{frame}



% ----- ab hier eigentlicher Inhalt -----------------------------------------
%Mal alter verwendbare Fragen gefunden ;) Erste muss noch erstetzt werden
\section[Einstieg]{Guten Morgen...}
\subsection*{}
\begin{frame}{Zum Warmwerden...}
	 Der Reguläre Ausdruck $a{b*}{c*}a$ ...
	   \begin{enumerate}
	   \item { \only<2->{ \color{green!50!black} }
	   ... beschreibt eine Typ-3 Sprache.
	   }
	   \item { \only<2->{ \color{red} }
	   ... beschreibt die Sprache $L_1=\{ab^nc^na | n \in \mathbb{N}_0\}$
	   }
	   \item { \only<2->{ \color{red} }
	   ... beschreibt eine endliche Sprache.
	   }
	   \end{enumerate}

	 Rechtslineare Grammatiken ...
	   \begin{enumerate}
	   \item { \only<3->{ \color{green!50!black} }
	   ... beschreiben reguläre Sprachen.
	   }
	   \item { \only<3->{ \color{green!50!black} }
	   ... dürfen $\epsilon$-Produktionen enthalten.
	   }
	   \item { \only<3->{ \color{green!50!black} }
	   ... lassen sich in einen endlichen Automaten überführen.
	   }
	   \end{enumerate}
\end{frame}

\begin{frame}{Zum Warmwerden...}
	 Für die Typen-Hierachie von Grammatiken gilt, ...
	   \begin{enumerate}
	   \item { \only<2->{ \color{red} }
	   ... alle Grammatiken lassen sich als regulärer Ausdruck angeben
	   }
	   \item { \only<2->{ \color{red} }
	   ... reguläre Sprachen sind die umfangreichsten.
	   }
	   \item { \only<2->{ \color{red} }
	   ... wenn die Grammatik $G$ Typ-2 ist (also eine KFG), so ist $L(G)$ auch Typ-2.
	   }
	   \end{enumerate}
	   
	   
	 Ein Akzeptor, bei dem der Startzustand auch akzeptierender Zustand ist ...
	   \begin{enumerate}
	   \item { \only<3>{ \color{red} }
	   ... akzeptiert alle Worte $w \in A $.
	   }
	   \item { \only<3>{ \color{green!50!black} }
	   ... akzeptiert immer auch das leere Wort $ \epsilon $.
	   }
	   \item { \only<3>{ \color{red} }
	   ... akzeptiert alle Worte $w \in A^* $.
	   }
	   \end{enumerate}
\end{frame}

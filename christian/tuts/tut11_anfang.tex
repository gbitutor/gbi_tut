
% ----- ab hier eigentlicher Inhalt -----------------------------------------
%Mal alter verwendbare Fragen gefunden ;) Erste muss noch erstetzt werden
\section[Einstieg]{Guten Morgen...}
\subsection*{}
\begin{frame}{Zum Warmwerden...}
  Eine formale Sprache $L$...
    \begin{enumerate}
    \item { \only<2->{ \color{green!50!black} }
    ... ist eine Menge von Wörtern
    }
    \item { \only<2->{ \color{red} }
    ... basiert immer auf einem endlichen Automaten $A$
    }
    \item { \only<2->{ \color{red} }
    ... kann gleich einem Wort $w$ sein
    }
    \end{enumerate}
	
  Eine formale Grammatik $G$...
    \begin{enumerate}
    \item { \only<3->{ \color{green!50!black} }
    ... lässt sich als Tupel (N,T,S,P) angeben.
    }
    \item { \only<3->{ \color{red} }
    ... erzeugt die endlich große Sprache $L(G)$.
    }
    \item { \only<3->{ \color{red} }
    ... ist immer kontextfrei und/oder regulär.
    }
    \end{enumerate}

	 Gegeben $G_2=(\{A,B\}, \{a,b\}, A, \{A \to Ab|Ba|a, B \to Aa|b \})$. $L(G_2)$ ...
	   \begin{enumerate}
	   \item { \only<4->{ \color{green!50!black} }
	   ... enthält unendlich viele Elemente.
	   }
	   \item { \only<4->{ \color{red} }
	   ... kann nicht durch eine KFG beschrieben werden.
	   }
	   \item { \only<4->{ \color{green!50!black} }
	   ... wird von einem endlichen Akzeptoren $A_2$ akzeptiert.
	   }
	   \end{enumerate}
\end{frame}

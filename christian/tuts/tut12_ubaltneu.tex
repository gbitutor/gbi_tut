

\section[Blatt 11]{Aufgabenblatt 11}
\subsection*{}
\begin{frame}
	\frametitle{Aufgabenblatt 11}
	\begin{block}{Blatt 11}
		\begin{itemize}
			\item Abgaben: 18 / 18
			\item Punkte: 11,1/20
		\end{itemize}
   \end{block}
	\begin{block}{Probleme}
 		\begin{itemize}
		   \item bei jedem Automaten Startzustand angeben \pause
		   \item zu jedem Akzeptor die akzeptierenden Zustände angeben
 	  \end{itemize}
	\end{block}
\end{frame}

\section[Blatt 12]{Aufgabenblatt 12}
\subsection*{}
\begin{frame}
	\frametitle{Aufgabenblatt 12}
	\begin{block}{Blatt 12}
		\begin{itemize}
			\item Abgabe: 25.01.2013 um 12:30 Uhr im Untergeschoss des Infobaus
			\item Punkte: maximal 20
		\end{itemize}
  	\end{block}
	\begin{block}{Themen}
		\begin{itemize}
	  		\item Strukturelle Induktion \pause
	  		\item Akzeptoren, reguläre Ausdrücke und rechtslineare Grammatiken \pause
	  		\item Turing Maschinen \pause
	  		\item Turing Maschinen bauen\pause 
	  		\item Turing Maschinen verstehen 
	 	\end{itemize}
	\end{block}
\end{frame}

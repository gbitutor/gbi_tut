
\section[Blatt 5]{Aufgabenblatt 5}
\subsection*{}
\begin{frame}
	\frametitle{Aufgabenblatt 5}
	\begin{block}{Blatt 5}
		\begin{itemize}
			\item Abgaben: 17 / 19
			\item Punkte: Durchschnitt 9,1 von 21
		\end{itemize}
  \end{block}
  
	\begin{block}{häufige Fehler...}
		\begin{itemize}
			\item[5.4:] achtet auf Paare und Mengennotation, lassen sich nicht beliebig mischen
		 \end{itemize}
	\end{block}
\end{frame}

\section[Blatt 6]{Aufgabenblatt 6}
\subsection*{}
\begin{frame}
	\frametitle{Aufgabenblatt 6}
	\begin{block}{Blatt 6}
		\begin{itemize}
			\item Abgabe: 30.11.2012 um 12:30 Uhr im Untergeschoss des Infobaus
			\item Punkte: maximal 20
		\end{itemize}
  \end{block}
	\begin{block}{Themen}
		\begin{itemize}
	  		\item Relationen
		\begin{itemize}
	  		\item Konkatenation
	  		\item Identität
	  \end{itemize}
	  		\item Homomorphismen
	  		\item Huffman-Codes
	  \end{itemize}
	\end{block}
\end{frame}

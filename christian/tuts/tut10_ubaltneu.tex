% !TeX root = ../tut10.tex
% !TeX encoding = latin1

\section[Rückblick Üb 9]{Aufgabenblatt 9}
\subsection*{}
\begin{frame}
	\frametitle{Aufgabenblatt 9}
	\begin{block}{Blatt 9}
		\begin{itemize}
			\item Abgaben: 14 / 19
			\item Punkte: Durchschnitt 6,1 von 20
		\end{itemize}
   \end{block}
	\begin{block}{Probleme}
 		\begin{itemize}
           \item 9.1: Die Laufzeitabschätzungen müssen für fast alle n gelten!
 	  \end{itemize}
	\end{block}
\end{frame}

\section[Blatt 10]{Aufgabenblatt 10}
\subsection*{}
\begin{frame}
	\frametitle{Aufgabenblatt 10}
	\begin{block}{Blatt 10}
		\begin{itemize}
			\item Abgabe: 11.01.2013 um 12:30 Uhr im Untergeschoss des Infobaus
			\item Punkte: maximal 20
		\end{itemize}
  	\end{block}
	\begin{block}{Themen}
		\begin{itemize}
        \item Rekursion
        \item Master-Theorem
	  		\item Endliche Automaten
	  		\item Mealy, Moore, endl. Akzeptoren
	 	\end{itemize}
	\end{block}
\end{frame}

\section[R�ckblick]{Aufgabenblatt 7}
\subsection*{}
\begin{frame}
	\frametitle{Aufgabenblatt 7}
	\begin{block}{Blatt 7}
		\begin{itemize}
			\item Abgaben: 20 / 24
			\item Punkte: Durchschnitt der abgegeben Bl�tter: 14,0 / 20
		\end{itemize}
   \end{block}
	\begin{block}{h�ufige Fehler}
 		\begin{itemize}
 	  		\item[7.1] beide Richtungen beachten \pause
 	  		\item[7.4 b)] in unger. Graphen muss Summe der Knotengrade gerade sein
 	  \end{itemize}
	\end{block}
\end{frame}

\section[Blatt 8]{Aufgabenblatt 8}
\subsection*{}
\begin{frame}
	\frametitle{Aufgabenblatt 8}
	\begin{block}{Blatt 8}
		\begin{itemize}
			\item Abgabe: 16.12.2011 um 12:30 Uhr im Untergeschoss des Infobaus
			\item Punkte: maximal 20
		\end{itemize}
  	\end{block}
	\begin{block}{Themen}
		\begin{itemize}
	  		\item Graphen
				\begin{itemize}
					\item Isomorphie
					\item Warshall-Algorithmus
					\item Wegematrix
				\end{itemize}
	 	\end{itemize}
	\end{block}
\end{frame}

% ----- ab hier eigentlicher Inhalt -------------------------------------------
\section[Einstieg]{Guten Morgen...}
\subsection*{}
\begin{frame}{Zum Warmwerden...}
  Die Menge der nat�rlichen Zahlen $\mathbb{N_+}$...
    \begin{enumerate}
    \item { \only<2->{ \color{red} }
    ... enth�lt die Null.
    }
    \item { \only<2->{ \color{green!50!black} }
    ... enth�lt nur nichtnegative ganze Zahlen.
    }
    \item { \only<2->{ \color{green!50!black} }
    ... ist eine Teilmenge der reelen Zahlen.
    }
    \end{enumerate}

  F�r zwei Funktionen f und g gilt...
    \begin{enumerate}
    \item { \only<3->{ \color{red} }
    ... ihre Konkatenation ist kommutativ.
    }
    \item { \only<3->{ \color{red} }
    ... der Werte- und Zielbereich sind stets gleich.
    }
    \item { \only<3->{ \color{green!50!black} }
    ... $f: x\mapsto x+1$ und $g: x \mapsto x+1$ k�nnen verschiedene Funktionen sein.
    }
    \end{enumerate}

%   F�r die Menge $\{ \epsilon \}$ gilt ...
%     \begin{enumerate}
%     \item { \only<4>{ \color{red} }
%     ... $|\{ \epsilon \}|=0$.
%     }
%     \item { \only<4>{ \color{green!50!black} }
%     ... ist Teilmenge von $A^*$ mit $A=\{a,b\}$.
%     }
%     \item { \only<4>{ \color{green!50!black} }
%     ... enth�lt die Menge der W�rter der L�nge $0$.
%     }
%     \end{enumerate}
\end{frame}

\section[Rückblick]{Aufgabenblätter}

\subsection{Aufgabenblatt 7}
\begin{frame}
	\frametitle{Aufgabenblatt 7}
	\begin{block}{Blatt 7}
		\begin{itemize}
			\item Abgaben: 18 / 19
			\item Punkte: Durchschnitt der abgegeben Blätter: $13,7$ / 20
		\end{itemize}
   \end{block}
	\begin{block}{häufige Fehler}
 		\begin{itemize}
 	  		\item[7.3] ein Beispiel ist kein Beweis
 	  \end{itemize}
	\end{block}
\end{frame}

\subsection{Aufgabenblatt 8}
\begin{frame}
	\frametitle{Aufgabenblatt 8}
	\begin{block}{Blatt 8}
		\begin{itemize}
			\item Abgaben: 19 / 19
			\item Punkte: Durchschnitt der abgegeben Blätter: $15,2$ / 19
		\end{itemize}
   \end{block}
	\begin{block}{häufige Fehler}
 		\begin{itemize}
 	  		\item[8.2c] wenn eine Begründung gefordert ist, gebt eine an \pause
 	  		\item[8.3] Warshall Alorithmus: $W_1$ enth. !nicht! einfach alle Wege der Länge 1
 	  \end{itemize}
	\end{block}
\end{frame}

\section[Blatt 9]{Aufgabenblatt 9}
\subsection*{}
\begin{frame}
	\frametitle{Aufgabenblatt 9}
	\begin{block}{Blatt 9}
		\begin{itemize}
			\item Abgabe: 21.12.2012 um 12:30 Uhr im Untergeschoss des Infobaus
			\item Punkte: maximal 20
		\end{itemize}
  	\end{block}
	\begin{block}{Themen}
		\begin{itemize}
			\item Effizienzklassen
			\item Effizienz von Algorithmen
	 	\end{itemize}
	\end{block}
\end{frame}

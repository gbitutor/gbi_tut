\section[Definitionen]{Definitionen}
\subsection*{}
\begin{frame}
\frametitle{Algorithmen}
	\begin{block}{Aus der Vorlesung}
	Algorithmen haben folgende Eigenschaften:
	\begin{itemize}
	  \item endliche Beschreibung (Wort �ber einem Alphabet)
	  \item elementare Aussagen (effektiv in einem Schritt ausf�hrbar)
	  \item Determinismus (n�chste Anweisung ist festgelegt)
	  \item endliche Eingabe errechnet endliche Ausgabe
	  \item endlich viele Schritte
	  \item funktioniert f�r beliebig gro�e Eingaben
	  \item nachvollziehbar/verst�ndlich
	\end{itemize}
	\end{block}
	\begin{block}{DIV und MOD}
	\begin{itemize}
	  \item $a$ $div$ $b$ ist das Ergebnis der ganzzahligen Division $a/b$
	  \item $a \mod b$ ist der Rest der ganzzahligen Division $a/b$
	\end{itemize}
	\end{block}
\end{frame}

\begin{frame}
 	\frametitle{Schleifen} 
        \begin{block}{Ziel}
        \begin{itemize}
        	\item Bestimmte Berechnungen sollen wiederholt ausgef�hrt werden, bis eine bestimmte Bedingung eintritt.
        	\item Dies wird u.a. durch eine for-Schleife (Z�hlschleife) erreicht.
        \end{itemize}
        \end{block}
	\pause
        \begin{block}{Syntax der for-Schleife in Pseudocode}
		for $i \leftarrow 0$ to $10$ do  //z�hlt von 0 bis 10\\
		     ... (Schleifeninhalt)
        \end{block}
	\pause
        \begin{block}{Syntax der for-Schleife in Java}
		for (int $i=0$; $i \leq 10$; i++)  //z�hlt von 0 bis 10\\
		     ... (Schleifeninhalt)
        \end{block}
\end{frame}

\begin{frame}
 	\frametitle{Schleifeninvariante}
	Als Schleifeninvariante werden Eigenschaften einer Schleife bezeichnet, die zu einem bestimmten Punkt bei jedem Durchlauf g�ltig sind, unabh�ngig von der Zahl ihrer derzeitigen Durchl�ufe. Typischerweise enthalten Schleifeninvarianten Wertebereiche von Variablen und Beziehungen der Variablen untereinander.\\
	\begin{block}{Sinn und Zweck}
	Schleifeninvarianten...
        \begin{itemize}
        	\item sind Aussagen, die am Anfang und am Ende eines Schleifendurchlaufes gelten
        	\item helfen, die Korrektheit eines Programmes zu beweisen
        	\item beweist man meist durch Induktion
        \end{itemize}		
	\end{block}
\end{frame}

\begin{frame}
 	\frametitle{Schleifeninvariante-Beispiel}
	\begin{block}{einfaches Beispiel}
		//Eingaben $a,b \in N_0$ \\
		$S \leftarrow a$ \\
		$Y \leftarrow b$ \\
		for $i \leftarrow 0$ to $b-1$ do \\
		$\qquad S \leftarrow S + 1$ \\
		$\qquad Y \leftarrow Y - 1$ \\
		od \\
		Output S //Ausgabe von S als Ergebnis
	\end{block}
	\begin{block}{Funktion}
		Was macht dieses Programm? \\
		\only<2->{Es berechnet die Summe von a und b.}
	\end{block}
\end{frame}

\begin{frame}
 	\frametitle{Schleifeninvariante Beispiel}
	\begin{block}{einfaches Beispiel}
		$S \leftarrow a$ \\
		$Y \leftarrow b$ \\
		for $i \leftarrow 0$ to $b-1$ do \\
		$\qquad S \leftarrow S + 1$ \\
		$\qquad Y \leftarrow Y - 1$ \\
		od \\
	\end{block}
	\begin{block}{Wertetabelle f�r $a=6$ und $b=4$}
		    \begin{tabular}{*{5}{>{$}r<{$}}}
		      & S & Y  \\
		       &  6 & 4    \\
		      i=0 &  7 & 3    \\
		      i=1 &  8 & 2 \\
		      i=2 &  9 & 1 \\
		      i=3 & 10 & 0
		    \end{tabular}
	\only<2->{$S+Y = a+b$}
	\end{block}
\end{frame}

\section[Suchalgorithmen]{Suchalgorithmen f�r W�rter}
\subsection*{}
\begin{frame}
	\frametitle{Suchalgorithmen}
	\begin{block}{Suchalgorithmen sind}
		Algorithmen, die etwas �ber das Vorkommen eines Zeichens $x \in A$ in einem Wort $w \in A$* aussagen.
	\end{block}
\end{frame}

\begin{frame}
	\frametitle{Algorithmenentwurf}
	\begin{block}{Aufgabe 1}
		Entwerfe einen Algorithmus, der berechnet, ob x in w vorkommt!
	\end{block}
	\visible<2->{
	\begin{block}{L�sung}
		$p \leftarrow -1$ \\
		for $i \leftarrow 0$ to $n-1$ do \\
		$p \leftarrow \begin {cases}
		1 & \textnormal { falls } w(i)=\textnormal{x}\\
		p & \textnormal { sonst} \end {cases}$
	\end{block}
	}
\end{frame}

\begin{frame}
	\frametitle{Algorithmenentwurf}
	\begin{block}{Aufgabe 2}
		Entwerfe einen Algorithmus, der die letzte Stelle im Wort, an der x in w vorkommt, berechnet!
	\end{block}
	\visible<2->{
	\begin{block}{L�sung}
		$p \leftarrow -1$ \\
		for $i \leftarrow 0$ to $n-1$ do \\
		$p \leftarrow \begin {cases}
		i & \textnormal { falls } w(i)=\textnormal{x}\\
		p & \textnormal { sonst} \end {cases}$
	\end{block}
	}
\end{frame}

\begin{frame}
	\frametitle{Algorithmenentwurf}
	\begin{block}{Aufgabe 3}
		Entwerfe einen Algorithmus, der die erste Stelle im Wort, an der x in w vorkommt, berechnet!
	\end{block}
	\visible<2->{
	\begin{block}{L�sung}
		$p \leftarrow -1$ \\
		for $i \leftarrow 0$ to $n-1$ do \\
		$p \leftarrow \begin {cases}
		i & \textnormal { falls } w(i)=\textnormal{x} \wedge p<0\\
		p & \textnormal { sonst} \end {cases}$
	\end{block}
	}

\end{frame}

\begin{frame}
	\frametitle{Korrektheitsbeweis}
	\begin{block}{Beweis der Korrektheit eines Programms (Aufgabe 2)}
		Durch Induktion wird gezeigt, dass nach den ersten k Schleifendurchl�ufen p die letzte Position von x in den ersten k Zeichen von w ist.
	\end{block}
	\visible<2->{
	\begin{block}{Beweis}
        \begin{itemize}
        	\item Induktionsanfang: k=0, p=-1 ist wahr.
        	\item Induktionsannahme: f�r ein festes $k<|w|$ gilt: Nach den ersten k Schleifendurchl�ufen ist p die Position des letzten x in den ersten k Zeichen von w.
        	\item Induktionsschritt: $k \rightarrow k+1$\\
		Wir betrachten den k+1ten Schleifendurchlauf, w�hrend dem das Zeichen w(k) betrachtet wird (2 F�lle!).
        \end{itemize}
	\end{block}
	}
\end{frame}


\begin{frame}
	\frametitle{Korrektheitsbeweis}
	\begin{block}{Fall 1: $w(k)=\textnormal{x}$}
        	Die Position des letzten x ist unter den ersten k+1 Zeichen jetzt die Position k+1. Nach Induktionsannahme gilt zu Beginn des Schleifendurchlaufs: p ist die letzte Position von x unter den ersten k Zeichen von w. Aufgrund des Programmes wird p nun k+1, so dass am Ende des k+1ten Schleifendurchlaufs gilt: p ist die Position des letzten x unter den ersten k+1 Zeichen von w.
	\end{block}

\end{frame}

\begin{frame}
	\frametitle{Korrektheitsbeweis}
	\begin{block}{Fall 2: $w(k)\ne \textnormal{x}$:}
        	Die letzte Position von x ist unter den ersten k+1 Zeichen gleich der letzten Position von x unter den ersten k Zeichen von w. Nach Induktionsannahme gilt zu Beginn des Schleifendurchlaufs: p ist die letzte Position von x unter den ersten k Zeichen von w. Aufgrund des Programmes bleibt p nun gleich, so dass am Ende des k+1ten Schleifendurchlaufs gilt: p ist die letzte Position von x unter den ersten k+1 Zeichen von w.
	\end{block}
	Damit ist die Behauptung gezeigt.

\end{frame}



\section[Palindromtest]{Palindromtest}
\subsection*{}
\begin{frame}
	\frametitle{Palindrome}
	\begin{block}{Was ist ein Palindrom?}
		Palindrome sind W�rter, die von vorne gelesen das gleiche Wort ergeben, wie von hinten gelesen.
	\end{block}
\end{frame}

\begin{frame}
	\frametitle{Palindromtest}
	\begin{block}{Aufgabe 5}
		Schreibe ein Programm, das f�r W�rter w untersucht, ob w ein Palindrom ist!
	\end{block}
	\visible<2->{
	\begin{block}{L�sung}
		$p \leftarrow 1$ \\
		for $i \leftarrow 0$ to $n-1$ do \\
		$p \leftarrow \begin {cases}
		p & \textnormal { falls } w(i)= w(n-1-i)\\
		-1 & \textnormal { sonst} \end {cases}$
	\end{block}
	}

\end{frame}

\section[Vollst. Ind.]{Vollst�ndige Induktion}
\subsection*{}
\begin{frame}
	\frametitle{Beweisverfahren der vollst�ndigen Induktion}
	\only<1-2>{
	\begin{block}{Die Theorie}
		Der Beweis erfolgt in folgenden Schritten:
		\begin{enumerate}
			\item Induktionsanfang: Die Aussage wird f�r $n=n_0$ gezeigt
			\item Induktionsvorraussetzung/-annahme: Die Aussage sei f�r \textbf{ein} $n$ wahr. 
			\item Induktionsschluss/-schritt: Aus dem Schluss von $n$ auf $n+1$ (in der Regel mit Hilfe der IV) folgt, dass die Aussage f�r alle nat�rlichen Zahlen $n>n_0$ gilt.
		\end{enumerate}
	\end{block}}
	\only<2-3>{
	\begin{block}{Ein Beispiel:}
	Beweise durch vollst�ndig Induktion $1+2+3+...+n=\frac{n*(n+1)}{2}$:
	\only<3>{
	\begin{itemize}
		\item[IA] $n=1$: $1=\frac{1*(1+1)}{2}=1$ ist erf�llt
		\item[IV]	$1+2+3+...+n=\frac{n*(n+1)}{2}$ gilt f�r ein $n \in \mathbb{N}$
		\item[IS] \begin{equation}
				\begin{split}
					 1+2&+3+...+(n+1) \nonumber\\
					&=1+2+3+...+n+(n+1) \nonumber\\
					&\overset{IV}{=} \frac{n*(n+1)}{2}+(n+1) \nonumber\\
					&=\frac{(n+1)*(n+2)+2(n+1)}{2}=\frac{n^2+3n+2}{2} \nonumber\\
					&=\frac{(n+1)*(n+2)}{2}
				\end{split}
			\end{equation}
	\end{itemize}}
	\end{block}}
\end{frame}

\subsection*{}
\begin{frame}
	\frametitle{Und weils so sch�n ist...}
	\begin{block}{Ihr schon wieder...}
	 Es sei $n \in \mathbf{N}$ und $a,b \in \mathbf{R}$. Beweist durch vollst�ndige Induktion:
  F�r $f(x) = e^{ax+b} $ gilt $f^{(n)}=a^n*e^{ax+b}$
	\end{block}
\end{frame}

\section{Abschluss}
% Studis anzuregen dar�ber nachzudenken, ob sie wirklich alles wissen, ansonsten nachlesen oder fragen nachtr�glich stellen, dann kann in der n�chsten Woche nochmal drauf eingegangen werden
\subsection*{}
\begin{frame}
	\frametitle{Zum Schluss...}
	\begin{block}{Was ihr nun wissen solltet!}
	\begin{itemize}
		\visible<2->{\item Was ist eine for-Schleife?}
		\visible<3->{\item Was ist eine Schleifeninvariante?}
		\visible<4->{\item Wie entwirft man einen Suchalgorithmus?}
		\visible<5->{\item Wie funktioniert ein Korrektheitsbeweis}
		\visible<6->{\item Was sind Palindrome? Vorgehen beim Algorithmenentwurf.}
	\end{itemize}
	\end{block}

	\visible<6->{
	\begin{block}{Ihr wisst was nicht?}
		Stellt \textbf{jetzt} Fragen!
	\end{block}}
\end{frame}

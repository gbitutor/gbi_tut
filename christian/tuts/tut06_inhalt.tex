% ----- ab hier eigentlicher Inhalt -------------------------------------------

\section[Relationen]{Relationen}
\subsection*{}
\begin{frame}
	\frametitle{Relationen -- Überblick}
	\begin{block}{Relationen anschaulich}
		Durch Relationen werden Elemente einer oder mehrerer Mengen in Beziehung zueinander gesetzt:
		\begin{itemize}
			\item praktisch jede alltägliche Aussage enthält Relationen \pause
			\item Beispiel: 'Das Haus hat vier Außenwände'
		\end{itemize}
	\end{block}
	\pause
	\begin{block}{Wozu brauchen wir das?}
		In der Informatik werden Relationen zur Modellierung von Systemen benötigt: \pause
		\begin{itemize}
			\item Relationen sind Grundlage der verschiedenen Diagramme der Unified
			Modeling Language \pause
			\item die graphische Darstellung von Relationen ergibt Graphen
		\end{itemize}
	\end{block}

\end{frame}

\begin{frame}
	\frametitle{Relationen mathematisch}
	\begin{block}{Definitionen - aus dem 1. Tutorium}
		\begin{itemize}
			\item Das karthesische Produkt zweier Mengen ist definiert als $A \times B := \left\{(a, b)|a \in A, b \in B\right\}$
			\item $R \subseteq A \times B$ heißt Relation
	  	\end{itemize}
	\end{block}
	\pause
	\begin{block}{Definition}
	\begin{itemize}
		\item Eine Relation $R$ bezieht sich auf zwei Grundmengen $M_1$,$M_2$ und es gilt $R \subseteq M_1\times M_2$. \pause
		\item Eine Relation $R$ heißt homogen, wenn $M_1=M_2$ gilt.
	\end{itemize}
	\end{block}
\end{frame}

%

\begin{frame}
	\frametitle{Relationen}
	\begin{block}{Definition: Produkt}
		Sind $R \subseteq M_1 \times M_2$ und $S \subseteq M_2 \times M_3$ zwei Relationen, dann heißt \pause
		\begin{itemize}
			\item $S \circ R = \{(x,z) \in M_1 \times M_3 \mid \exists y \in M_2 : (x,y) \in R \wedge (y,z) \in S \} $ das {\em Produkt der Relationen S und R} \pause
			\item $Id_M = \{(x,x) \mid x \in M \}$ heißt die {\em identische Abbildung}
		\end{itemize}
	\end{block}
	\pause
	\begin{block}{Definition: Potenz}
		Sei $R \subseteq M \times M$ eine {\em binäre} Relation, dann heißt
		\begin{itemize}
			\item $R^i$ {\em die i-te Potenz} von R und ist definiert als: \pause
			\begin{itemize}
				\item $R^0=Id_M$ \pause
				\item $\forall i \in \mathbb N_0: R^{i+1}=R \circ R^i$
			\end{itemize}
		\end{itemize}
	\end{block}
\end{frame}

\section[Reflexiv-transitive Hülle]{Reflexiv-transitive Hülle}
\subsection*{}
\begin{frame}
	\frametitle{Reflexiv-transitive Hülle}

	\begin{block}{mögliche Attribute homogener Relationen}
		\begin{description}
			\item[reflexiv] $x R x$
			\item[transitiv] Aus $x R y$ und $y R z$ folgt $x R z$
			\item[symmetrisch] Aus $x R y$ folgt $y R x$
		\end{description} \pause
		Gelten alle diese Eigenschaften, handelt es sich um eine Äquivalenzrelation.
	\end{block}
		\pause
	\begin{block}{Definition}
		Die sogenannte reflexiv-transitive Hülle einer Relation R ist
		\begin{itemize}
			\item $R^* = \bigcup_{i=0}^{\infty} R^i$
		\end{itemize}
	\pause
		Sie ist die Erweiterung der Relation um die Paare, die notwendig sind um Reflexivität und Transitivität herzustellen.
	\end{block}

\end{frame}

\begin{frame}
	\frametitle{Reflexiv-transitive Hülle}
	\begin{block}{Anschauliches Beispiel: StudiVZ}
		\begin{itemize}
			\item $R \subseteq M \times M $ sei die "`ist-befreundet-mit"'-Relation.
			\item $M = \{ Gertrud, Holger, Lars, Katja, Martin, Nina \} $ %alphabetische Anfänge == leichter merkbar
			\item $R = \{ (Martin,Holger), (Lars,Katja), (Nina,Holger),$ \\
						$(Gertrud,Holger), (Katja, Nina) \} \bigcup \{${dazu sym. Tupel}$\}$ \pause
			\item dann ist $R^0=\{ (Martin,Martin), ..., (Holger,Holger) \}$
			\item und $R^1=R$ und \pause
			\item $R^2=\{ (Martin,Nina), (Martin,Gertrud), (Martin,Martin),$ \\
						$(Lars,Nina), (Lars,Lars), (Nina,Gertrud),(Nina,Martin),$ \\
						$(Nina,Nina), (Nina,Lars), (Katja,Katja), (Katja,Holger), $ \\
						$(Gertrud,Gertrud), (Gertrud,Martin), (Gertrud,Nina), $ \\
						$(Holger,Holger), (Holger,Katja)\}$
			\item $R^*=$ ? \pause Ist $R^*$ eine Äquivalenzrelation?
			%\item wegen Symmetrie und Unsinnigkeit der Reflexivität, entfällt Einiges % Funktionen sind kommutativ, Relationen symmetrisch
		\end{itemize}
	\end{block}
\end{frame}


\subsection*{}
\begin{frame}
	\frametitle{Relationen graphisch}
	\begin{block}{Ihr seid dran...}
		\begin{enumerate}
			\item Überlegt euch, wie eine Relation graphisch aussehen könnte. Zeigt ein Beispiel mit mindestens $4$ verschiedenen Elementen %Grundmenge?
			\item Wie sieht nun graphisch die reflexiv-transitive Hülle aus?
		\end{enumerate}
	\end{block} \pause
    \begin{block}{mögliche Darstellung}
    	\begin{itemize}
          \item Relation als Pfeile von Element zu Element
          \item Relation als Matrix, d.h. wenn xRy ist Feld [x,y] $== 1$
        \end{itemize}
    \end{block}
\end{frame}


\section{Zahlensysteme}

\subsection*{}	%Von letztem Jahr info1
\begin{frame}
	\frametitle{Zahlensysteme umrechnen}
	\only<1>{
	\begin{block}{Was ist das?}
		\begin{itemize}
          \item Wir verwenden normalerweise das Dezimalsystem mit den Ziffern 0 bis 9
          \item Es gibt aber noch weitere Zahlensysteme, wie das Dualsysteme (mit den Ziffern 0 und 1)
          \item Hexadezimalsystem (mit den Ziffern von 0-9 und den Buchstaben A-F)
        \end{itemize}
	\end{block}

	\begin{block}{Darstellung}
		Eine Darstellung einer Zahl im Dualsystem ist wie folgt aufgebaut:
		$z_m z_{m-1} \ldots z_0, z_{-1} \ldots z_{-n}$ mit $(m,n \in \mathbb{N}_0 z_i \in \{ 0, 1 \} )$

	\end{block}}
	\only<2>{
	\begin{block}{0-9}
\begin{center}
			\begin{tabular}{c|c|c|c|c|c|c|c|c|c|c}
			Dez & 0 & 1 & 2 & 3 & 4 & 5 & 6 & 7 & 8 & 9 \\
			Bin & 0 & 1 & 10 & 11 & 100 & 101 & 110 & 111 & 1000 & 1001\\
			Oct & 0 & 1 & 2 & 3 & 4 & 5 & 6 & 7 & 10 & 11 \\
			Hex & 0 & 1 & 2 & 3 & 4 & 5 & 6 & 7 & 8 & 9
			\end{tabular}
\end{center}
    \end{block}
  	\begin{block}{10-15}
\begin{center}

			\begin{tabular}{c|c|c|c|c|c|c}
			Dez & 10 & 11 & 12 & 13 & 14 & 15 \\
			Bin & 1010 & 1011 & 1100 & 1101 & 1110 & 1111 \\
			Oct & 12 & 13 & 14 & 15 & 16 & 17 \\
			Hex & A & B & C & D & E & F
			\end{tabular}
\end{center}
    \end{block}
	}

\end{frame}

\begin{frame}
	\frametitle{Zahlensysteme}
	\begin{block}{Umrechnung}
		\only<3->{\scriptsize}
		Wert einer Dualzahl im Dezimalsystem:
\begin{center}
	$Z=\sum_{i=-n}^{m} z_i * 2^i$
\end{center} \pause
		Wert einer ganzzahligen Dezimalzahl $z$ im Dualsystem:
			\begin{enumerate}
				\item Finde das größte $n$ mit $2^n \leq z$
				\item Notiere 1, setze $z=z-2^n$ und setze $i=n-1$ .
				\item Teste, ob $2^{i} \leq z$
					\begin{itemize}
						\only<3->{\scriptsize}
						\item Wenn ja, dann notiere 1, setze $z=z-2^i$ und setze $i=i-1$
						\item Wenn nein, dann notiere 0 und setze $i=i-1$
					\end{itemize}
				\item Wiederhole Schritt 3 solange bis i=0
			\end{enumerate}
	\end{block}

	\only<3->{
	\begin{block}{Ihr seid dran}
		%\small
		Wandle $4242_{10}$ ins Dual-, Oktal- und Hexadezimalsystem um. \\
		Ein kleiner Tipp: 4 Stellen im Dualsystem lassen sich zu einer Stelle im Hexadezimalssystem zusammenfassen. $(0001 0001)_2=(11)_{16}$
	\end{block}}
\end{frame}

\subsection*{}
\begin{frame}{Ihr seid dran...}
	\begin{block}{Was macht der Algorithmus?}
		 \only<2->{//Eingabe: $w \in Z_2^*$\\}
    $x \leftarrow 0$ \\
    \only<3->{\color{red}$v\leftarrow\epsilon$ \color{black}\\}
    \textbf{for} $i\leftarrow 0$ to $|w| -1$ do \\
    \ \ $x \leftarrow 2x + num_2(w(i))$ \\
    \only<3->{\color{red} \ \ $v\leftarrow v \cdot w(i)$ \color{black} \\}
    od
    \visible<2->{//am Ende: $x=Num_2(w)$ \ \only<3->{$\land \ v=w$}}
  \end{block}
  \begin{block}{Analyse}
  	\begin{itemize}
	   	\only<1,2>{	\item Was macht diese Algorithmus? Was sind wohl die Ein- und Ausgaben? }
  		\item Was ist eine mögliche Schleifeninvariante? \\
  			TIPP: Ihr könnt den Code auch erweitern um eine geeignete Invariante zu finden. \\
  			\only<3->{Lsg.: $x = Num_2(v)$}
  	\end{itemize}
 \end{block}
\end{frame}


\section{Alphabete}
\subsection*{}


\begin{frame}{Aus der Vorlesung...}
	Warum macht man Übersetzungen? \pause
 \begin{itemize}
 \item \textbf{Lesbarkeit:} \pause Manchmal führen Übersetzungen zu kürzeren und
   besser lesbaren Texten.\\
   	 $A3$ ist leichter erfassbar als
   $10100011$
 \item \textbf{Kompression:}  \pause Manchmal führen Übersetzungen zu kürzeren
   Texten, die weniger Platz benötigen. Und zwar \emph{ohne} zu
   einem größeren Alphabet überzugehen.
 \item \textbf{Verschlüsselung:} \pause Manchmal will man Texte für andere unleserlich machen
 \item \textbf{Fehlererkennung} und \textbf{Fehlerkorrektur:} \pause Man kann Texte durch Übersetzung derart länger machen, dass man Fehler erkennen
 	oder diese sogar beheben kann
 \end{itemize}
\end{frame}


\section{Einschub}
\subsection{Homomorphismen}
% ------sollten wir in ein zwei einfache Sätze zusammenfassen, um die angst vor dem begriff zu nehmen-----------------------------------------------------------------
\begin{frame}{Homomorphismen}
\begin{block}{Definition}

Ein \emph{Homomorphismus} $h:A^* \rightarrow B^*$ ist eine Abbildung, die
durch die Funktionswerte $h(x)$ für alle $x\in A$ eindeutig festgelegt ist. \\ \pause
Insbesondere bleibt das neutrale Element das neutrale Element:

	\begin{align*}
    h(\epsilon) &= \epsilon \\
  	   	  h(wx) &= h(w) h(x)
	\end{align*}
	
	weiterhin wird die zugrundeliegende Struktur erhalten
	     \begin{itemize}
       \item auf $\mathbb{N}_0$ ist Verdoppelung Homomorphismus, Struktur der Addition bleibt erhalten
       \item auf Strings ist $upper()$ ein Homomorphismus
     \end{itemize}
\end{block}
\end{frame}


\subsection{Graphen}
\begin{frame}{Graphen}

 	\begin{block}{Bäume - Binärbäume}
     In der Regel\dots
     \begin{itemize}
       \item hat jeder Baum eine Wurzel und jeder Knoten maximal zwei Kinder/Nachfolger
       \item wird die Wurzel oben dargestellt
     \end{itemize}
	\end{block}
\end{frame}

\section{Huffman-Codes}
\subsection*{}
\begin{frame}
	\frametitle{Aus der Vorlesung:}
	\begin{block}{Wozu Huffman Codes?}
    		\begin{itemize}
              \item Huffman-Codes komprimieren ein Wort $w \in A$* indem \pause
              \item häufigere Symbole durch kürzere
              \item und seltener vorkommende Symbole durch längere Wörter
              kodiert werden \pause
              \item statt einzelnen Symbolen können auch längere Blöcke als
              kleinste Einheit gewählt werden
            \end{itemize}
    	\end{block}
\end{frame}

\begin{frame}{Huffman-Code}
	\begin{block}{Vorgehensweise}
		zwei Schritte:
    		\begin{enumerate}
			\item Konstruktion eines Baumes:
				\begin{itemize}
					\visible<2->{\item Blätter entsprechen $x \in A$}
					\visible<3->{\item Innere Knoten entsprechen Mengen von Symbolen}
					\visible<4->{\item An jedem Blatt wird das Symbol $x$ und dessen Häufigkeit notiert}
					\visible<5->{\item die zwei Elemente mit der geringsten Häufigkeit werden zu einem Elternknoten zusammengefasst}
				\end{itemize}
			\item Beschriftung der Kanten: links mit 0, rechts mit 1
		\end{enumerate}
		Der kürzeste Weg von der Wurzel zum Blatt gibt die Kodierung des jeweiligen Zeichens (im Blatt) an.
    	\end{block}
\end{frame}

\begin{frame}{Aufgabe}
	\begin{block}{Aufgabe 1}
    		Gegeben sei das Alphabet $X=\{a,b,c,d,e,f,g,h\}$.
    		\begin{enumerate}
			\item 1. Fall: Jedes Zeichen kommt genau einmal vor \\ Erstelle den Huffman-Code-Baum.\\
				\pause Wie lange wird die Kodierung von $w=badcfehg$? \pause
			\item 2. Fall: Zeichen a und b kommen zweimal, c viermal, d 8-mal, e 16-mal, f 32-mal, g 64-mal und h 128-mal vor. \\ Erstelle den Huffman-Code-Baum. Wie lange wird die Kodierung von $w=badcafehg$? \pause
			\item Wie lange wird ein Wort mit zweiter Zeichenverteilung, wenn man es mit dem ersten Code codiert?
			\item Wie lange wird ein Wort mit erster Zeichenverteilung, wenn man es mit dem zweiten Code codiert?
		\end{enumerate}
    	\end{block}

\end{frame}

\begin{frame}{Aufgabe}
	\begin{block}{Aufgabe 2}
    		Gegeben sei das Alphabet $X=\{a,b,c,d,e,f,g\}$ und die Auftrittswahrscheinlichkeiten $p(a)=\frac{3}{10}$, $p(b)=\frac{1}{10}$, $p(c)=\frac{1}{10}$, $p(d)=\frac{1}{7}$, $p(e)=\frac{1}{7}$, $p(f)=\frac{1}{7}$ und $p(g)=\frac{1}{14}$.
    		\begin{itemize}
			\item Erzeuge einen Huffman-Code C.
		\end{itemize}
    	\end{block}
	\visible<2->{
	\begin{block}{Lösung 2}
    		\begin{tabular}{cccccccc}
			Zeichen:&  a&    b&    c&    d&    e&    f&    g\\
			Wahrscheinlichkeit:&$\frac{3}{10}$&$\frac{1}{10}$&$\frac{1}{10}$&$\frac{1}{7}$&$\frac{1}{7}$&$\frac{1}{7}$&$\frac{1}{14}$ \\
			Code:&     00&  101&  110&  010&  011&  100&  111
		\end{tabular}
    	\end{block}
	}
\end{frame}

\begin{frame}{viele Codes}
	\begin{block}{mehrdeutig?}
    \begin{itemize}
      \item im Allgemeinen sind Huffman-Codes nicht indeutig:
		\item es können mehrere Zeichen gleichhäufig vorkommen
		\item Außerdem ist nicht festgelegt, welcher Knoten linker Nachfolger und welcher rechter Nachfolger eines inneren Knotens wird

		\item[$\Rightarrow$] Huffman-Codes sind nicht eindeutig
		\item Das macht aber nichts: alle, die sich für ein Wort $w$ ergeben können, sind "`gleich gut"'
	\end{itemize}
	\end{block}

\end{frame}


\section{Abschluss}
% Studis anzuregen darüber nachzudenken, ob sie wirklich alles wissen, ansonsten nachlesen oder fragen nachträglich stellen, dann kann in der nächsten Woche nochmal drauf eingegangen werden
\subsection*{}
\begin{frame}
	\frametitle{Zum Schluss...}
	\begin{block}{Was ihr nun wissen solltet!}
	\begin{itemize}
		\visible<2->{\item Was bedeutet Konkatenation von Relationen?}
		\visible<3->{\item Was tut ein Homomorphismus?}
	\end{itemize}
	\end{block}

	\visible<4->{
	\begin{block}{Ihr wisst was nicht?}
		Stellt \textbf{jetzt} Fragen!
	\end{block}}
\end{frame}

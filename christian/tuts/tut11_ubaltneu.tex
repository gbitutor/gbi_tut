
\section[R�ckblick �B 10]{Aufgabenblatt 10}
\subsection*{}
\begin{frame}
	\frametitle{Aufgabenblatt 10}
	\begin{block}{Blatt 10}
		\begin{itemize}
			\item Abgaben: 20 / 24
			\item Punkte: Durchschnitt 15,5 von 23
		\end{itemize}
   \end{block}
	\begin{block}{Probleme}
 		\begin{itemize}
			\item 10.3. Kante zum Startzustand nicht vergessen
 	  \end{itemize}
	\end{block}
\end{frame}

\section[Blatt 11]{Aufgabenblatt 11}
\subsection*{}
\begin{frame}
	\frametitle{Aufgabenblatt 11}
	\begin{block}{Blatt 11}
		\begin{itemize}
			\item Abgabe: 20.01.2012 um 12:30 Uhr im Untergeschoss des Infobaus
			\item Punkte: maximal 20
		\end{itemize}
  	\end{block}
	\begin{block}{Themen}
		\begin{itemize}
	  		\item Endliche Automaten
	  		\item Mealy, Moore, endl. Akzeptoren
			\item regul�re Ausdr�cke
	 	\end{itemize}
	\end{block}
\end{frame}

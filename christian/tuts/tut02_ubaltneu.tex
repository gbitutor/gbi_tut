
\section[Blatt 1]{Aufgabenblatt 1}
\subsection*{}
\begin{frame}
	\frametitle{Ein Blick zurück}
	\begin{block}{etwas Statistik}
		\begin{itemize}
			\item 19 Abgaben, weiter so!
			\item durchschnittliche Punktzahl: $13,3$/$19$ Punkten 
		\end{itemize}
	\end{block}

	\begin{block}{häufige Fehler...}
		\begin{itemize}
			\item[1.2:] Beispiele sind keine Beweise (anders als Gegenbeispiele) \pause
			\item[] äquivalente Ausdrücke sind nicht $=$, besser $\equiv$ \pause
			\item[] Behauptung markieren mit ``\texttt{z.z.:}'' oder ``$\overset{!}{=}$'' \pause
			\item[] Elemente, Paare und Mengen richtig notieren \pause
			\item[] richtige Notation von Mengen / Aussagen / Funktionen vermeidet Fehler \pause
			\item[] Aussagenlogische Definition von injektiv/surjektiv klausurrelevant
		 \end{itemize}
	\end{block}
\end{frame}

\section[Blatt 2]{Aufgabenblatt 2}
\subsection*{}
\begin{frame}
        \frametitle{Aufgabenblatt 2}
        \begin{block}{Blatt 2}
					\begin{itemize}
						\item Abgabe: 02.11.2012 um 12:30 Uhr im Untergeschoss des Infobaus
						\item Punkte: maximal 20
					\end{itemize}
        \end{block}
        \begin{block}{Themen}
        \begin{itemize}
        \item Prädikatenlogik 
        	\item Vollständige Induktion
        \end{itemize}
        \end{block}
\end{frame}

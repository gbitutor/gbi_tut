\section[Sprachen]{Formale Sprachen}
\begin{frame}
	\frametitle{Formale Sprachen}
	\begin{block}{Was ist das eigentlich?} \pause
		\begin{itemize}
          \item Eine formale Sprache $L$ ist eine Menge von Wörtern die aus einem beliebigen Alphabet $A$ erzeugt werden können. \pause
          \item $L$ soll stets alle (in einem bestimmten Sinn) korrekten Gebilde enthalten und alle nicht korrekten nicht.
        \end{itemize}
	\end{block}
 \pause
	\begin{block}{Ein Beispiel...}
		Die Sprache die alle Zustände einer Ampel beschreibt enthält \texttt{Grün} oder \texttt{Rot-Gelb} aber nicht die Phase \texttt{Grün-Rot}.
	\end{block}
\end{frame}	
	
\begin{frame}
	\frametitle{Jetzt wirds theoretisch...}
	\begin{block}{ein paar Definitionen}
  \begin{itemize}
	  \item formale Sprache: $L\subseteq A^*$
	  \item Produkt: $L_1\cdot L_2 = \{w_1w_2 \mid w_1\in L_1 \land w_2\in L_2\}$
	  	\begin{itemize}
	  		\item $\{a,bb\}\cdot \{aa,b\}=$ \only<2->{$\{aaa,ab,bbaa,bbb\}$}
	  		\item $L \cdot \{\epsilon \} =$ \only<3->{$L$}
	  	\end{itemize}
	  \visible<4->{	
	  \item Potenzen: $L^0=\{\epsilon \}$ und $L^{i+1}=L^i\cdot L$
	  \item Konkatenationsabschluss: \pause
	    \begin{equation*}
	      L^+ = \bigcup_{i=1}^{\infty} L^i \text{\qquad und \qquad}
	      L^* = \bigcup_{i=0}^{\infty} L^i
	    \end{equation*}}
	  \end{itemize}
  \end{block}
\end{frame}

\subsection*{}
\begin{frame}
	\frametitle{einige Beispiele}
	\only<1-3>{
	\begin{block}{Die Zahlen vom Typ \texttt{int}}
		Gebt eine formale Sprache $L_I$ aller legalen Zahlen vom Typ \texttt{int} an. \\
 		\visible<2->{
 		$A=\{0,\ldots,9\}$ \\
 		$L_I=$ \only<2>{ $A^*$ } \visible<3->{$\{\epsilon,-\}A^+$.} \\
 		\only<3>{Seid ihr mit der Lösung einverstanden?}
		}
	\end{block}}
	\only<4-7>{
	\begin{block}{Variabelnamen in \texttt{Java}}
		Gebt eine formale Sprache $L_V$ aller legalen Variablenamen in \texttt{Java} an. \\
	\end{block}
	\begin{block}{Lösung}
		\only<5-6>{
	    $A=\{\_ ,a,\ldots,z,A,\ldots,Z\}$, \\
      $B=A \cup \{0,\ldots,9\}$ \\
      $L_V= A \cdot B^*$. \\
    	Was fehlt?
    	\visible<6>{
    		\begin{itemize}
    			\item Umlaute
    			\item Schlüsselwörter sind als Variablenamen verboten
    		\end{itemize}
    	}
		}
		\only<7>{
		\hspace{1em}
			Also besser: \\
      $A=\{\_ ,a,\ldots,z,A,\ldots,Z, $ä$, $ö$, $ü$\}$, \\
      $B=A \cup \{0,\ldots,9\}$ \\
      $L_V= (A \cdot B^*) \smallsetminus\{\texttt{if}, \texttt{class}, \dots\}$
		}
	\end{block}
	\only<8->{
	\begin{block}{ohne \texttt{ab}}
	 	\begin{itemize}
           \item Gebt die formale Sprache $L$ aller Wörter über $A=\{a,b\}$an, in denen nirgends das Teilwort \texttt{ab} vorkommt.
           \item \visible<9-10>{Anders formuliert: In den erlaubten Wörtern müssen, wenn überhaupt, erst alle \texttt{b} kommen und danach, wenn überhaupt alle \texttt{a}}
         \end{itemize}

	 \visible<10>{$L=\{b\}^*\{a\}^*$}
	\end{block}
}}

\end{frame}

\subsection*{}
\begin{frame}
	\frametitle{noch einige Hinweise...}
	\begin{block}{Wörter \& Sprache}
		Wörter und Sprachen sind nicht das Gleiche! \\
		So ist $\texttt{abb}$ ist etwas anderes als $\{\texttt{abb}\}$. \\
		Und $\{\texttt{abb}\}^*$ gibt es, aber $\texttt{abb}^*$ kennen wir \textbf{nicht}.
	\end{block}
	
	\begin{block}{$L_1 L_2$}
		$L_1=\{a^n \mid n \in \mathbb{N}_0\}$ und  $L_2=\{b^n \mid n \in \mathbb{N}_0\}$ \\     				\textbf{Achtung:} $L_1L_2=\{a^k b^m \mid k\in\mathbb{N}_0\land m\in\mathbb{N}_0\}$  die Exponenten können verschieden sein!
	\end{block}
\end{frame}

\section[Mengen]{Mengenlehre}

\subsection*{}
\begin{frame}
	\frametitle{Mengenlehre}
	\begin{block}{Ergänzungen}
		\begin{itemize}
			\item Was ist eine \textbf{Mengendifferenz}?
			\only<2->{\item Sei $A:=\left\{ 1,2,3 \right\}$ und $B:=\left\{ 2,4,6 \right\}$ Was ist dann $A\setminus B$?} 
			\only<3->{\item $A$ ohne $B$, d.h. $A\setminus B = \left\{1,3\right\}$}  
			\only<4->{\item Was ist bei $A\cup B$ zu beachten?} 
			\only<5->{\item In einer Menge kommt ein Element \textbf{nie mehrfach} vor (und die Reihenfolge ist ohne Bedeutung).}
		\end{itemize}	
	\end{block}
\end{frame}

\subsection*{}
\begin{frame}
	\frametitle{Mengengleichheit}
	\begin{block}{Wie beweist man das nochmal?}
	%hier Pause um Diskussion aufkommen zu lassen
	\visible<2->{ Indem man beweist, dass $\subseteq$ und $\supseteq$ gelten}
	\end{block}
	\only<3->{
	\begin{block}{Beweise $L^* \cdot L = L^+$}
	  \begin{itemize}
	    \item $\subseteq$: \\
	    \visible<4->{
	      Wenn $w\in L^*\cdot L$, dann $w=w'w''$ mit
	      $w'\in L^*$ und $w''\in L$. \\
	      Also existiert ein $i\in\mathbb{N}_0$ mit $w'\in L^i$.\\
	      Also $w=w'w''\in L^i\cdot L = L^{i+1}$. \\
	      Da $i+1\in\mathbb{N}_+$, ist $L^{i+1}\subseteq L^+$, also $w\in L^+$.}
	    \item $\supseteq$: 
	    \visible<5->{	%den Teil die Studis probieren(!!!!) alleine lösen zu lassen
	    	Wenn $w\in L^+$, dann existiert ein $i\in\mathbb{N}_+$
	      mit $w\in L^i$. \\
	      Da $i\in \mathbb{N}_+$ ist $i=j+1$ für ein $j\in\mathbb{N}_0$, \\
	      also ist für ein $j\in\mathbb{N}_0$: $w\in L^{j+1}= L^j\cdot L$. \\
	      also $w=w'w''$ mit $w'\in L^j$ und $w''\in L$ . \\
	      Wegen $L^j\subseteq L^*$ ist $w=w'w''\in L^*\cdot L$.}
	   \end{itemize}
	
	\end{block}
	}
\end{frame}

\section{Abschluss}
% Studis anzuregen darüber nachzudenken, ob sie wirklich alles wissen, ansonsten nachlesen oder fragen nachträglich stellen, dann kann in der nächsten Woche nochmal drauf eingegangen werden
\subsection*{}
\begin{frame}
	\frametitle{Zum Schluss...}
	\begin{block}{Was ihr nun wissen solltet!}
	\begin{itemize}
		\visible<2->{\item Wie beweise ich Mengengleichheit?}
		\visible<3->{\item Was ist das Beweisverfahren der vollständigen Induktion?}
		\visible<4->{\item Was kann ich alles tolles damit anstellen?}
		\visible<5->{\item Wie kann ich meinen Tutor bei der Korrektur meines Übungsblattes positiv beeinflussen?}
	\end{itemize}
	\end{block}

	\visible<6->{
	\begin{block}{Ihr wisst was nicht?}
		Stellt \textbf{jetzt} Fragen!
	\end{block}}
\end{frame}

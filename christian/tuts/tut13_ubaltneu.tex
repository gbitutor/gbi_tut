\section[Rückblick]{Aufgabenblatt 12}
\subsection*{}
\begin{frame}
	\frametitle{Aufgabenblatt 12}
	\begin{block}{Blatt 12}
		\begin{itemize}
			\item Abgaben: 10 / 19
			\item Punkte: Durchschnitt 8,25 von 19
		\end{itemize}
		häufige Fehler:
		\begin{itemize}
			\item 3) Vorgehensweise der TM in eigenen Worten heißt: \\
			grob die Positionswechsel und Schreibvorgänge beschreiben, aber nicht einzelne Kanten des TM-Graphen oder Einträge der TM-Tabelle beschreiben
		\end{itemize}
   \end{block}
\end{frame}

\section[Blatt 13]{Aufgabenblatt 13}
\subsection*{}
\begin{frame}
	\frametitle{Aufgabenblatt 13}
	\begin{block}{Blatt 13}
		\begin{itemize}
			\item Abgabe: 01.02.2013 um 12:30 Uhr im Untergeschoss des Infobaus
			\item Punkte: maximal 22
		\end{itemize}
  	\end{block}
	\begin{block}{Themen}
		\begin{itemize}
	  		\item Akzeptoren
	  		\item Äquivalenzrelationen
	  		\item Nerode Relationen
	 	\end{itemize}
	\end{block}
\end{frame}

\section{Probeklausur}
\subsection*{}
\begin{frame}
	\frametitle{Probeklausur}
	\begin{block}{Termin}
		\begin{itemize}
			\item Freitag den 01.02.2013 anstelle der Gbi-Übung
			\item Ort: je nach Matrikelnummer im Audimax oder im HS -101 oder -102 im Geb. 50.34
      \item Teilnahme freiwillig
      \item Tutoren korrigieren die Abgaben, Rückgabe nächste Woche im Tutorium
      \item Ergebnis dient nur eurer Selbsteinschätzung
      \item Aufgaben werden von Tutoren erstellt, keine Garantie auf identische Aufgaben in der Klausur
		\end{itemize}
  	\end{block}
\end{frame}
